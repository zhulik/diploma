\newpage

\begin{center}
Федеральное государственное бюджетное образовательное учреждение\\
высшего профессионального образования \\
\vspace{0.5cm}
УФИМСКИЙ ГОСУДАРСТВЕННЫЙ АВИАЦИОННЫЙ ТЕХНИЧЕСКИЙ УНИВЕРСИТЕТ\\
\vspace{0.5cm}
ФАКУЛЬТЕТ  ИНФОРМАТИКИ  И  РОБОТОТЕХНИКИ \\
\vspace{0.5cm}
КАФЕДРА  ВЫЧИСЛИТЕЛЬНОЙ  МАТЕМАТИКИ  И  КИБЕРНЕТИКИ\\
\vspace{1cm}
\end{center}
\vspace{0.5cm}
\begin{flushright}
"УТВЕРЖДАЮ" \\
Зав. кафедрой  ВМК, д.т.н., проф.\\
\vspace{0.5cm}
\underline{\hspace{5cm}} Н.И. Юсупова\\
«08»    октября  2015  г.
\end{flushright}
\vspace{1cm}

\begin{center}
  \Large{ ЗАДАНИЕ }
\vspace{0.5cm}

на подготовку выпускной квалификационной работы
\end{center}

студента Синявского Глеба Николаевича

\begin{enumerate}
  \item Тема работы - Программа для предупреждения пересечения стволов скважин\\
  ( утверждена  распоряжением  по  факультету  №  72/1    от “  15 октября   2015  г )
  \item Срок  представления  работы  “20”     января     2016  г.
  \item Описание задачи\\
    Необходимо разработать программный продукт, позволяющий усреднять и визуализировать замеры стволов скважин, а так же
    позволять оценивать расстояния между ними.
  \item Математическая часть\\
  В качестве математических моделей, принятых для реализации в рамках программного продукта,
  необходимо использовать уравнения для расчета координат ствола методом среднего угла.
  \item Спецификация входных и выходных данных\\
  Входные данные - csv-файлы, содержащие результаты замера ствола скважины. Выходные - визуализация скважины в пространстве, визуализации оценки расстояний
  между стволами.
  \item Применяемые инструментальные средства\\
  Библиотека построение графического интерфейса - Qt. СУБД - SQLite. Библиотека\\визуализации - MathGL.
  \item Особые условия эксплуатации программного продукта\\
  Основная ОС для запуска программного продукта - Windows 7 и старше, но продукт должен разрабатываться как кроссплатформенный и иметь возможность запуска
  под управлением ОС Linux.
  \item Дополнительные условия\\
  Продукт должен иметь возможность импортировать csv произвольного формата, для этого должен быть разработан мастер импорта, позволяющий
  выбирать диапазон ячеек таблицы и указывать их тип.
\end{enumerate}

\vspace{\fill}

Руководитель работы \underline{\hspace{5cm}}

Консультант \underline{\hspace{6.5cm}}
\begin{center}
  Дата  выдачи   «08»  октября       2015 г.
\end{center}
\newpage
