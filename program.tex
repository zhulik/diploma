\newpage
\section{Программное обеспечение}

\subsection{Аналитический обзор существующих программных технологий, применимых при решении поставленных задач}

Существует два основных подхода к разработке программного обеспечения: процедурный и объектно-ориентированный.
Каждый из них имеет свои преимущества и недостатки.

При использовании процедурного подхода, обычно появляется огромное количество глобальных переменных, и структурирование
происходит на уровне модулей (файлов).

Объектно-ориентированное или объектное программирование — парадигма программирования, в которой основными концепциями
являются понятия объектов и классов.

Класс — это тип, описывающий устройство объектов. Понятие «класс» подразумевает некоторое поведение и способ представления.
Понятие «объект» подразумевает нечто, что обладает определённым поведением и способом представления. Говорят, что объект — это экземпляр класса. Класс можно сравнить с чертежом, согласно которому создаются объекты. Обычно классы разрабатывают таким образом, чтобы их объекты соответствовали объектам предметной области.

Класс является описываемой на языке терминологии (пространства имён) исходного кода моделью ещё не существующей сущности,
т. е. объекта.

Объект — сущность в адресном пространстве вычислительной системы, появляющаяся при создании экземпляра класса (например,
после запуска результатов компиляции (и связывания) исходного кода на выполнение).

Перечислим некоторые достоинства ООП:
\begin{itemize}
  \item классы позволяют проводить конструирование из полезных компонент, обладающих простыми инструментами,
    что дает возможность абстрагироваться от деталей реализации.
  \item данные и операции вместе образуют определенную сущность и они не «размазываются» по всей программе,
    как это нередко бывает в случае процедурного программирования.
  \item локализация кода и данных улучшает наглядность и удобство сопровождения программного обеспечения.
  \item инкапсуляция информации защищает наиболее критичные данные от несанкционированного доступа.
\end{itemize}

Model-view-controller (MVC, «модель-представление-контроллер», «модель-вид-контроллер») — схема использования нескольких
шаблонов проектирования, с помощью которых модель приложения, пользовательский интерфейс и взаимодействие с пользователем
разделены на три отдельных компонента таким образом, чтобы модификация одного из компонентов оказывала минимальное
воздействие на остальные. Данная схема проектирования часто используется для построения архитектурного каркаса,
когда переходят от теории к реализации в конкретной предметной области.

Предметно-ориентированное проектирование (реже проблемно-ориентированное, англ. Domain-driven design, DDD) —
это набор принципов и схем, помогающих разработчикам создавать изящные системы объектов. При правильном применении
оно приводит к созданию программных абстракций, которые называются моделями предметных областей. В эти модели входит
сложная бизнес-логика, устраняющая промежуток между реальными условиями области применения продукта и кодом.

\subsection{Архитектура разрабатываемого программного продукта}

Для создания программного продукта был применен объектно-ориентированный подход к программированию,
т.к. решение данной задачи требует жесткого контроля над пользователем при работе с системы.
Объектно-ориентированный подход применялся там где, происходит сбор информации т.к. необходимо четко определить формат данных.

Qt использует концепцию MVC для отделения данных(Model) от их представления(View) в таких компонентах, как QTreeView, QListView и
QTableView \cite{qt}.

\subsection{Язык программирования и инструментальные средства разработки}

\subsubsection{Язык C++}
На данный момент, C++ остаётся одним из самых популярных и производительных языков программирования и применяется практически во всех прикладных областях
программирования, от низкоуровневого программирования для микроконтроллеров, до высокопроизводительных серверных приложений и компьютерных игр.

\subsubsection{SQLite}
SQLite — это встраиваемая кроссплатформенная СУБД, которая поддерживает достаточно полный набор команд SQL и доступна в исходных кодах (на языке C). На данный
момент является самой популярной встраиваемой СУБД. Применяется как на персональный компьютерах, так и в мобильных ОС и "умных" телевизорах.

\subsubsection{Qt}
Qt — кроссплатформенный инструментарий разработки ПО на языке программирования C++, доступен в исходных текстах. Позволяет создавать кроссплатформенные приложения с богатыми возможностями
графического интерфейса, работой с сетью, мультимедиа, БД и 3D-графикой. В окружении каждой поддерживаемой ОС будет выглядеть максимально похоже на "родные" приложения
системы.

\subsubsection{MathGL}
MathGL — кроссплатформенная библиотека для визуализации данных. Имеет интеграцию с Qt.

\subsubsection{Обоснованность выбора технологий}
На данный момент указанные технологии являются единственным способом, как выполнить требования о кроссплатформенности, так и получить лёгкий в поддержке
продукт, базирующийся на надёжных и поддерживаемых библиотеках.

\subsection{Описание структуры программного продукта}
Структура программного продукта изображения на \fullref{schemes/architecture.png}

\putimage[1]{schemes/architecture.png} {
  Структура программного продукта
}

\subsection{Описание интерфейса пользователя}
После запуска программы открывается основное окно, главными элементами которого являются дерево, представляющее иерархию предметной отрасли,
и таблица редактирования выбранного в дереве уровня, \fullref{manual/10.jpg}

\putimage[1]{manual/10.jpg} {
  Главное окно программы
}

Интерфейс пользователя построен с использованием стандартных элементов и выглядит аналогично на всех поддерживаемых платформах. Вверху окна находится
главное меню, в котором скрыты настройки, переключения языка и другие редко используемые функции. Под главным меню располагается панель инструментов,
на которой расположены более популярные функции "График", "Импорт", "Предупреждение пересечения стволов" и "Создать усредненный".

На \fullref{manual/19.jpg} изображен диалог с профилем ствола, изображение можно вращать мышью и настраивать отображение с помощью настроек справа.

\putimage[1]{manual/19.jpg} {
  Диалог профиля ствола
}

Подробно интерфейс пользователя описан в приложении "Руководство пользователя".
