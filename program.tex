\newpage
\section{Программное обеспечение}

\subsection{Аналитический обзор существующих программных технологий, применимых при решении поставленных задач}
\subsection{Архитектура разрабатываемого программного продукта}
\subsection{Язык программирования и инструментальные средства разработки}

\subsubsection{Язык C++}
На данный момент, C++ остаётся одним из самых популярных и производительных языков программирования и применяется практически во всех прикладных областях
программирования, от низкоуровневого программирования для микроконтроллеров, до высокопроизводительных серверных приложений и компьютерных игр.

\subsubsection{SQLite}
SQLite — это встраиваемая кроссплатформенная СУБД, которая поддерживает достаточно полный набор команд SQL и доступна в исходных кодах (на языке C). На данный
момент является самой популярной встраиваемой СУБД. Применяется как на персональный компьютерах, так и в мобильных ОС и "умных" телевизорах.

\subsubsection{Qt}
Qt — кроссплатформенный инструментарий разработки ПО на языке программирования C++, доступен в исходных текстах. Позволяет создавать кроссплатформенные приложения с богатыми возможностями
графического интерфейса, работой с сетью, мультимедиа, БД и 3D-графикой. В окружении каждой поддерживаемой ОС будет выглядеть максимально похоже на "родные" приложения
системы.

\subsubsection{MathGL}
MathGL — кроссплатформенная библиотека для визуализации данных. Имеет интеграцию с Qt.

\subsubsection{Обоснованность выбора технологий}
На данный момент указанные технологии являются единственным способом, как выполнить требования о кроссплатформенности, так и получить лёгкий в поддержке
продукт, базирующийся на надёжных и поддерживаемых библиотеках.

\subsection{Технологии разработки ПО (моделирование разработки ПО, управление разработкой ПО, конфигурирование ПО, технологии тестирования ПО)}
\subsection{Описание структуры программного продукта}
\subsection{Описание интерфейса пользователя}
