\textbf{Метрическая оценка качества программного продукта.}

В данной части дипломной работы проводится оценка качества программного продукта согласно ГОСТ 28195-89.

\textbf{Определение подкласса программных средств}

Данное программное средство относится к подклассу 509 – Прочие ПС.

\textbf{Показатели надёжности программного средства}

\begin{ztable}{|p{2cm}|p{10cm}|p{2cm}|p{2cm}|}{ Оценочные элементы фактора “Надёжность ПС”}{reliability_table}
    \hline
    Код элемента & Наименование & Метод оценки & Оценка\\

    \endhead

    \hline
    \multicolumn{4}{|c|}{ \textbf{Средства восстановления при ошибках на входе}} \\

    \hline
    Н0101 & Наличие требований к программе по устойчивости функционирования при наличии ошибок во входных данных  & Экспертный & 1 \\

    \hline
    Н0102 & Возможность обработки ошибочных ситуаций & То же & 1 \\

    \hline
    Н0103 & Полнота обработки ошибочных ситуаций & » & 1 \\

    \hline
    Н0104 & Наличие тестов для проверки допустимых значений входных данных & » с& 0 \\

    \hline
    Н0105 & Наличие системы контроля полноты входных данных & » & 0 \\

    \hline
    Н0106 & Наличие средств контроля корректности входных данных & » & 1 \\

    \hline
    Н0107 & Наличие средств контроля непротиворечивости входных данных & » & 0 \\

    \hline
    Н0108 & Наличие проверки параметров и адресов по диапазону их значений & » & 1 \\
    Н0109 & Наличие обработки граничных результатов & » & 1 \\

    \hline
    Н0110 & Наличие обработки неопределенностей & » & 0,6 \\

    \hline
    &&& 0,8 \\

    \hline
    \multicolumn{4}{|c|}{ \textbf{Средства восстановления при сбоях оборудования}} \\

    \hline
    Н0201 & Наличие требований к программе по восстановлению процесса выполнения в случае сбоя операционной системы, процессора, внешних устройств & » & 0 \\

    \hline
    Н0202 & Наличие требований к программе по восстановлению результатов при отказах процессора, ОС & » & 1 \\

    \hline
    Н0203 & Наличие средств восстановления процесса в случае сбоев оборудования & » & 0 \\

    \hline
    Н0204 & Наличие возможности разделения по времени выполнения отдельных функций программ & » & 1 \\

    \hline
    Н0205 & Наличие возможности повторного старта с точки останова & » & 1 \\

    \hline
    &&& 0,6 \\

    \hline
    \multicolumn{4}{|c|}{ \textbf{Реализация управления средствами восстановления}} \\

    \hline
    Н0301  & Наличие централизованного управления процессами, конкурирующими из-за ресурсов & » & 1 \\

    \hline
    Н0302  & Наличие возможности автоматически обходить ошибочные ситуации в процессе вычисления & » & 0 \\

    \hline
    Н0303 & Наличие средств, обеспечивающих завершение процесса решения в случае помех & » & 1 \\

    \hline
    Н0304  & Наличие средств, обеспечивающих выполнение программы в сокращенном объеме в случае ошибок или помех & » & 0 \\

    \hline
    Н0305 & Показатель устойчивости к искажаемым воздействиям& Расчетный & 0 \\

    \hline
    &&& 0,4 \\

    \hline
    \multicolumn{4}{|c|}{ \textbf{Функционирование в заданных режимах}} \\

    \hline
    Н0401 & Вероятность безотказной работы & То же & 1 \\

    \hline
    \multicolumn{4}{|c|}{ \textbf{Обеспечение обработки заданного объема информации}} \\

    \hline
    Н0501 & Оценка по среднему времени восстановления & » & 1 \\

    \hline
    Н0502 & Оценка по продолжительности преобразования входного набора данных в выходной & » & 1 \\

    \hline
    &&& 1 \\
    \hline
\end{ztable}

\textbf{Показатели сопровож­дения}

\begin{ztable}{|p{2cm}|p{10cm}|p{2cm}|p{2cm}|}{Оценочные элементы фактора “Сопровождаемость ПС”}{support_table}
    \hline
    Код элемента & Наименование & Метод оценки & Оценка\\

    \endhead

    \hline
    \multicolumn{4}{|c|}{\textbf{Простота архитектуры проекта}} \\

    \hline
    С0101 & Наличие модульной схе­мы программы & Экспертный & 1 \\

    \hline
    С0102 & Оценка программы по числу уникальных модулей & » & 1 \\

    &&& 1 \\

    \hline
    \multicolumn{4}{|c|}{\textbf{Сложность архитектуры проекта}} \\

    \hline
    С0201 & Наличие ограничений на размеры модуля & » & 0 \\

    \hline
    \multicolumn{4}{|c|}{\textbf{Межмодульные связи}} \\


    \hline
    С030 & Наличие требований к не­зависимости модулей про­граммы от типов и форма­тов выходных данных & » & 0 \\

    \hline
    С0301& Наличие проверки кор­ректности передаваемых данных & » & 1 \\

    \hline
    С0302 & Оценка простоты прог­раммы по числу точек вхо­да и выхода  & Расчетный & 0,01 \\

    \hline
    С0303 & Осуществляется ли пере­дача результатов работы модуля через вызывающий его модуль  & Экспертный & 1 \\

    \hline
    С0304& Осуществляется ли конт­роль за правильностью дан­ных, поступающих в вызы­вающий модуль от вызыва­емого & » & 1 \\



    \hline
    &&& 0,6 \\

    \hline
    \multicolumn{4}{|c|}{\textbf{Экспертиза принятой системы идентификации}} \\

    \hline
    С0601 & Использование при пост­роении программ метода структурного программиро­вания & » & 1 \\

    \hline
    С0602& Соблюдение принципа разработки программы сверху вниз & » & 1 \\

    \hline
    С0603 & Оценка программы по числу циклов с одним вхо­дом и одним выходом & » & 1 \\

    \hline
    С0604 & Оценка программы по числу циклов & » & 1 \\

    \hline
    &&& 1 \\

    \hline
    \multicolumn{4}{|c|}{\textbf{Комментарии логики программ проекта}} \\


    \hline
    С0801 & Наличие комментариев ко всем машинозависимым час­тям программы & » & 0 \\

    \hline
    С0802 & Наличие комментариев к машинозависимым операто­рам программы & » & 0 \\

    \hline
    С0803 & Наличие комментариев в точках входа и выхода про­граммы & » & 1 \\

    \hline
    &&& 0,3 \\

    \hline
    \multicolumn{4}{|c|}{\textbf{Оформление текста программ}} \\

    \hline
    С0901 & Соответствие комментари­ев принятым соглашениям & » & 0 \\

    \hline
    С0902 & Наличие комментариев-за­головков программы с ука­занием ее структурных и функциональных характе­ристик & » & 0 \\

    \hline
    С0903& Оценка ясности и точнос­ти описания последователь­ности функционирования всех элементов программы& » & 0 \\

    \hline
    &&& 0 \\

    \hline
    \multicolumn{4}{|c|}{\textbf{Простота кодирования}} \\


    \hline
    С1001 & Используется ли язык высокого уровня & » & 1 \\

    \hline
    С1002 & Оценка простоты прог­раммы по числу переходов по условию & Расчетный & 0,3 \\

    \hline
    &&& 0,6 \\

    \hline
\end{ztable}

\textbf{Показатели удобства применения}

\begin{ztable}{|p{2cm}|p{10cm}|p{2cm}|p{2cm}|}{Оценочные элементы фактора “Удобство применения ПС”}{convenience_table}
    \hline
    Код элемента & Наименование & Метод оценки & Оценка\\

    \endhead

    \hline
    \multicolumn{4}{|c|}{\textbf{Освоение работы ПС}} \\


    \hline
    У0101 & Возможность освоения програм­мных средств по документации & Экспертный & 1 \\

    \hline
    У0102 & Возможность освоения ПС на конт­рольном  примере при  помощи  ЭВМ & То же & 1 \\

    \hline
    У0103 & Возможность поэтапного освоения ПС & » & 1 \\

    \hline
    &&& 1 \\

    \hline
    \multicolumn{4}{|c|}{\textbf{Документация для освоения}} \\

    \hline
    У0201 & Полнота и понятность документа­ции для освоения & » & 1 \\

    \hline
    У0202 & Точность документации для освое­ния & » & 1 \\

    \hline
    У0203 & Техническое исполнение докумен­тации & » & 0,4 \\

    \hline
    &&& 0,8 \\


    \hline
    \multicolumn{4}{|c|}{\textbf{Полнота пользовательской документации}} \\


    \hline
    У0301 & Наличие краткой аннотации & » & 1 \\

    \hline
    У0302 & Наличие описания решаемых задач & » & 1 \\

    \hline
    У0303 & Наличие описания структуры  функ­ции ПС & » & 1 \\

    \hline
    У0304  & Наличие описания основных функ­ций ПС & » & 1 \\

    \hline
    У0306  & Наличие описания частных функ­ций & » & 1 \\

    \hline
    У0307  & Наличие описания алгоритмов & » & 0 \\

    \hline
    У0308  & Наличие описания межмодульных интерфейсов & » & 0 \\

    \hline
    У0309  & Наличие описания пользовательских интерфейсов & » & 1 \\

    \hline
    У0310  & Наличие описания входных и вы­ходных данных & » & 1 \\

    \hline
    У0311 & Наличие описания диагностических сообщений & » & 0 \\

    \hline
    У0312  & Наличие описания основных харак­теристик ПС & » & 1 \\

    \hline
    У0314  & Наличие описания программной среды функционирования ПС & » & 1 \\

    \hline
    У0315  & Достаточность документации для ввода ПС в эксплуатацию & » & 1 \\

    \hline
    У0316 & Наличие информации технологии переноса  для  мобильных программ & » & 0 \\

    \hline
    &&& 0,7 \\

    \hline
    \multicolumn{4}{|c|}{\textbf{Точность пользовательской документации}} \\

    \hline
    У0401& Соответствие оглавления содержа­нию документации & » & 1 \\

    \hline
    У0402& Оценка оформления документации & » & 1 \\

    \hline
    У0403& Грамматическая правильность из­ложения документации & » & 1 \\

    \hline
    У0404& Отсутствие противоречий & » & 1 \\

    \hline
    У0405& Отсутствие неправильных ссылок & » & 1 \\

    \hline
    У0406& Ясность формулировок и описаний & » & 1 \\

    \hline
    У0407& Отсутствие неоднозначных форму­лировок и описаний & » & 1 \\

    \hline
    У0408& Правильность использования тер­минов & » & 1 \\

    \hline
    У0409 & Краткость,  отсутствие лишней  де­тализации & » & 1 \\

    \hline
    У0410 & Единство формулировок & » & 1 \\

    \hline
    У0411 & Единство обозначений & » & 1 \\

    \hline
    У0412 & Отсутствие ненужных повторений & » & 1 \\

    \hline
    У0413 & Наличие нужных объяснений & » & 1 \\

    \hline
    &&& 1 \\

    \hline
    \multicolumn{4}{|c|}{\textbf{Понятность пользовательской документации}} \\

    \hline
    У0501  & Оценка стиля изложения  & » & 1 \\

    \hline
    У0502  & Дидактическая разделенность & » & 1 \\

    \hline
    У0503  & Формальная разделенность & » & 1 \\

    \hline
    У0504  & Ясность логической структуры  & » & 1 \\

    \hline
    У0505  & Соблюдение стандартов и правил изложения в документации  & » & 1 \\

    \hline
    У0506 & Оценка по числу ссылок вперед в тексте документов  & » & 0 \\

    \hline
    &&& 0,8 \\

    \hline
    \multicolumn{4}{|c|}{\textbf{Техническое исполнение пользовательской документации}} \\

    \hline
    У0601 & Наличие оглавления & » & 1 \\

    \hline
    У0602 & Наличие предметного указателя & » & 0 \\

    \hline
    У0603 & Наличие перекрестных ссылок & » & 0 \\

    \hline
    У0604 & Наличие всех требуемых разделов & » & 1 \\

    \hline
    У0605& Соблюдение непрерывности нуме­рации страниц документов & » & 1 \\

    \hline
    У0606 & Отсутствие незаконченных разделов абзацев, предложений & » & 1 \\

    \hline
    У0607 & Наличие всех рисунков, чертежей, формул, таблиц & » & 1 \\

    \hline
    У0608 & Наличие всех строк и примечаний & » & 1 \\

    \hline
    У0609 & Логический порядок частей внутри главы & » & 1 \\

    \hline
    &&& 0,8 \\

    \hline
    \multicolumn{4}{|c|}{\textbf{Прослеживание вариантов пользовательской документации}} \\

    \hline
    У0701 & Наличие полного перечня докумен­тации & » & 1 \\

    \multicolumn{4}{|c|}{\textbf{Эксплуатация}} \\

    \hline
    У0801 & Уровень языка общения пользова­теля с программой & » & 1 \\

    \hline
    У0802  & Легкость и быстрота загрузки и запуска программы & » & 1 \\

    \hline
    У0803 & Легкость и быстрота завершения работы программы & » & 1 \\

    \hline
    У0804  & Возможность распечатки содержи­мого программы & » & 0,7 \\

    \hline
    У0805  & Возможность приостанови и пов­торного запуска работы без потерь информации & » & 1 \\

    \hline
    &&& 0,9 \\

    \hline
    \multicolumn{4}{|c|}{\textbf{Управление меню}} \\


    \hline
    У0901 & Соответствие меню требованиям пользователя & » & 1 \\

    \hline
    У0902 & Возможность прямого перехода вверх и вниз  по многоуровневому ме­ню (пропуск уровней) & » & 1 \\

    \hline
    &&& 1 \\



    \hline
    \multicolumn{4}{|c|}{\textbf{Функция Help}} \\

    \hline
    У1001 & Возможность управления подроб­ностью  получаемых выходных дан­ных  & » & 1 \\

    \hline
    У1002 & Достаточность полученной инфор­мации для продолжения работы  & » & 1 \\

    \hline
    &&& 1 \\

    \hline
    \multicolumn{4}{|c|}{\textbf{Управление данными}} \\

    \hline
    У1101 & Обеспечение удобства ввода дан­ных & » & 1 \\

    \hline
    У1102 & Легкость восприятия & » & 1 \\

    \hline
    &&& 1 \\

    \hline
    \multicolumn{4}{|c|}{\textbf{Рабочие процедуры}} \\

    \hline
    У1201 & Обеспечение программой выполне­ния предусмотренных рабочих про­цедур & » & 1 \\

    \hline
    У1202 & Достаточность информации, выда­ваемой программой для составления дополнительных процедур & » & 1 \\

    \hline
    &&& 1 \\

    \hline
\end{ztable}

\textbf{Показатели эффективности}

\begin{ztable}{|p{2cm}|p{10cm}|p{2cm}|p{2cm}|}{Оценочные элементы фактора “Эффективность ПС”}{efficiency_table}
    \hline
    Код элемента & Наименование & Метод оценки & Оценка\\

    \endhead

    \hline
    \multicolumn{4}{|c|}{ \textbf{Уровень автоматизации}} \\

    \hline
    Э0101 & Проблемно-ориентированные функции & Экспертный & 1 \\

    \hline
    Э0102 & Машинно-ориентированные функции & То же & 1 \\

    \hline
    Э0103 & Функции ведения и управления & » & 1 \\

    \hline
    Э0104 & Функции ввода/вывода & » & 1 \\

    \hline
    Э0105 & Функции защиты и проверки данных & » & 0 \\

    \hline
    Э0106 & Функции защиты от несанкционированного доступа & » & 1 \\

    \hline
    Э0107 & Функции контроля доступа & » & 1 \\

    \hline
    Э0108 & Функции защиты от внесения изменений & » & 1 \\

    \hline
    Э0109 & Наличие соответствующих границ функциональных областей & » & 1 \\

    \hline
    Э0110 & Число знаков после запятой в результатах вычислений & » & 1 \\

    \hline
    &&& 0,9 \\

    \hline
    \multicolumn{4}{|c|}{ \textbf{Временная эффективность}} \\

    \hline
    Э0201 & Время выполнения программ & » & 1 \\

    \hline
    Э0202 & Время реакции и ответов & » & 1 \\

    \hline
    Э0203 & Время подготовки & » & 1 \\

    \hline
    Э0205 & Затраты времени на защиту данных  & » & 0 \\

    \hline
    Э0206 & Время компиляции  & » & 1 \\

    \hline
    &&& 0,8 \\

    \hline
    \multicolumn{4}{|c|}{ \textbf{Ресурсоемкость}} \\


    \hline
    Э0301 & Требуемый объем  внутренней  памяти  & » & 1 \\

    \hline
    Э0302 & Требуемый объем  внешней  памяти  & » & 1 \\

    \hline
    Э0303 & Требуемые периферийные устройства  & » & 1 \\

    \hline
    Э0304 & Требуемое базовое программное обеспечение  & » & 1 \\

    \hline
    &&& 1 \\


    \hline
\end{ztable}

\textbf{Показатели универсальности}

\begin{ztable}{|p{2cm}|p{10cm}|p{2cm}|p{2cm}|}{Оценочные элементы фактора “Универсальность ПС”}{versality_table}
    \hline
    Код элемента & Наименование & Метод оценки & Оценка\\

    \endhead

    \hline
    \multicolumn{4}{|c|}{\textbf{Зависимость от используемого комплекса технических средств}} \\

    \hline
    Г0701 & Оценка зависимости программ от ёмкости оперативной памяти ЭВМ  & » & 1 \\

    \hline
    Г0702 & Оценка зависимости временных характеристик программы от скорости вычислений ЭВМ  & » & 1 \\

    \hline
    Г0703 & Оценка зависимости функционирования программы от числа внешних запоминающих устройств и их общей емкости  & » & 0 \\

    \hline
    Г0704 & Оценка зависимости функционирования программы от специальных устройств ввода-вывода & » & 1 \\

    \hline
    &&& 0,7 \\

    \hline
    \multicolumn{4}{|c|}{\textbf{Зависимость от базового программного обеспечения}} \\

    \hline
    Г0801 & Применение специальных языков программирования & » & 1 \\

    \hline
    Г0802 & Оценка зависимости программы от программ  операционной системы  & » & 1 \\

    \hline
    Г0803 & Зависимость от других программных средств & » & 1 \\

    \hline
    &&& 1 \\

    \hline
    \multicolumn{4}{|c|}{\textbf{Изоляция немобильности}} \\

    \hline
    Г0901 & Оценка локализации непереносимой части программы & » & 1 \\

    \hline
    &&& 1 \\

    \hline
    \multicolumn{4}{|c|}{\textbf{Простота кодирования}} \\

    \hline
    Г1001& Оценка использования отрицательных или булевых выражений & » & 1 \\

    \hline
    Г1002 & Оценка программы по использованию условных переходов & » & 1 \\

    \hline
    Г1003& Оценка программы по использованию безусловных переходов  & » & 0 \\

    \hline
    Г1004 & Оформление процедур входа и выхода из циклов  & » & 1 \\

    \hline
    Г1005 & Ограничения на модификацию переменной индексации в цикле  & » & 1 \\

    \hline
    Г1007 & Оценка программы по использованию локальных переменных  & » & 1 \\

    \hline
    Г1006 & Оценка модулей по направлению потока управления & » & 0 \\

    \hline
    &&& 0,7 \\

    \hline
    \multicolumn{4}{|c|}{\textbf{Число комментариев}} \\

    \hline
    Г1101 & Оценка программы по числу комментариев & » & 1 \\

    \hline
    &&& 1 \\

    \hline
    \multicolumn{4}{|c|}{\textbf{Качество комментариев}} \\

    \hline
    Г1201 & Наличие заголовка в программе  & » & 1 \\

    \hline
    Г1202& Комментарии к точкам ветвлений  & » & 1 \\

    \hline
    Г1203 & Комментарии к машинозависимым частям программы & » & 1 \\

    \hline
    Г1204 & Комментарии к машинозависимым операторам программы  & » & 0 \\

    \hline
    Г1205 & Комментарии к операторам объявления переменных  & » & 1 \\

    \hline
    Г1206 & Оценка семантики операторов  & » & 1 \\

    \hline
    Г1207 & Наличие соглашений по форме представления комментариев  & » & 0 \\

    \hline
    Г1208 & Наличие общих комментариев к программам & » & 1 \\

    \hline
    &&& 0,7 \\

    \hline
    \multicolumn{4}{|c|}{\textbf{Использование описательных средств языка}} \\

    \hline
    Г1301 & Использование языков высокого уровня & » & 1 \\

    \hline
    Г1302& Семантика имен используемых переменных& » & 1 \\

    \hline
    Г1303 & Использование отступов, сдвигов и пропусков при формировании текста & » & 1 \\

    \hline
    Г1304& Размещение операторов по строкам & » & 1 \\

    \hline
    &&& 1 \\

    \hline
    \multicolumn{4}{|c|}{\textbf{Независимость модулей}} \\


    \hline
    Г1401 & Передача информации для управления по параметрам & » & 1 \\

    \hline
    Г1402 & Наличие передачи результатов работы между модулями & » & 1 \\

    \hline
    Г1403 & Наличие проверки правильности данных, получаемых модулями от вызываемого модуля & » & 1 \\

    \hline
    Г1404 & Использование общих областей памяти & » & 1 \\

    \hline
    Г1405 & Параметрическая передача входных данных & » & 1 \\


    \hline
    &&& 1 \\


    \hline
\end{ztable}

\textbf{Показатели корректности}

\begin{ztable}{|p{2cm}|p{10cm}|p{2cm}|p{2cm}|}{Оценочные элементы фактора “Корректность ПС”}{correctness_table}
    \hline
    Код элемента & Наименование & Метод оценки & Оценка\\

    \endhead

    \hline
    \multicolumn{4}{|c|}{ \textbf{Требования, предъявляемые к полноте документации разработчика}} \\

    \hline
    К0101 & Наличие всех необходимых документов для понимания и использования ПС & Экспертный & 1 \\

    \hline
    К0102 & Наличие описания и схемы иерархии модулей программы & » & 1 \\

    \hline
    К0103& Наличие описания основных функций & » & 1 \\

    \hline
    К0104 & Наличие описания частных функций & » & 1 \\

    \hline
    К0105 & Наличие описания данных & » & 1 \\

    \hline
    К0106 & Наличие описания алгоритмов & » & 1 \\

    \hline
    К0107 & Наличие описания интерфейсов между модулями & » & 1 \\

    \hline
    К0108 & Наличие описания интерфейсов  с пользователями & » & 1 \\

    \hline
    К0109 & Наличие описания используемых числовых методов & » & 0 \\

    \hline
    К0110& Указаны ли все численные методы & » & 0 \\

    \hline
    К0111& Наличие описания всех параметров & » & 0 \\

    \hline
    К0112 & Наличие описания методов настройки системы & » & 1 \\

    \hline
    К0113 & Наличие описания всех  диагностических сообщений & » & 1 \\

    \hline
    К0114 & Наличие описания способов проверки работоспособности программы & » & 1 \\

    \hline
    &&& 0,8 \\



    \hline
    \multicolumn{4}{|c|}{ \textbf{Полнота программной документации}} \\

    \hline
    К0201 & Реализация всех исходных модулей & » & 1 \\

    \hline
    К0202 & Реализация всех основных функций & » & 1 \\

    \hline
    К0203 & Реализация всех частных  функций & » & 1 \\

    \hline
    К0204 & Реализация всех алгоритмов & » & 1 \\

    \hline
    К0205 & Реализация всех взаимосвязей в системе & » & 1 \\

    \hline
    К0206 & Реализация всех интерфейсов между модулями & » & 1 \\

    \hline
    К0207 & Реализация возможности настройки системы & » & 1 \\

    \hline
    К0208 & Реализация диагностики всех граничных и аварийных ситуаций & » & 1 \\

    \hline
    К0209 & Наличие определения всех данных (переменные, индексы, массивы и проч.) & » & 1 \\

    \hline
    К0210 & Наличие интерфейсов с пользователем & » & 1 \\

    \hline
    &&& 1 \\



    \hline
    \multicolumn{4}{|c|}{ \textbf{Непротиворечивость документации разработчика}} \\


    \hline
    К0301  & Отсутствие противоречий в описании частных функций & » & 1 \\

    \hline
    К0302 & Отсутствие противоречий в описании основных функций в разных документах & » & 1 \\

    \hline
    К0303 & Отсутствие противоречий в описании алгоритмов & » & 1 \\

    \hline
    К0304  & Отсутствие противоречий в описании взаимосвязей в системе & » & 1 \\

    \hline
    К0305 & Отсутствие противоречий в описании интерфейсов между модулями & » & 1 \\

    \hline
    К0306  & Отсутствие противоречий в описании интерфейсов с пользователем & » & 1 \\

    \hline
    К0307  & Отсутствие противоречий в описании настройки системы & » & 1 \\

    \hline
    К0309  & Отсутствие противоречий в описании иерархической структуры сообщений & » & 1 \\

    \hline
    К0310 & Отсутствие противоречий в описании диагностических сообщений & » & 1 \\

    \hline
    К0311 & Отсутствие противоречий в описании данных & » & 1 \\


    \hline
    &&& 1 \\



    \hline
    \multicolumn{4}{|c|}{ \textbf{Непротиворечивость программы}} \\

    \hline
    К0401 & Отсутствие противоречий в выполнении основных функций & » & 1 \\

    \hline
    К0402 & Отсутствие противоречий в выполнении частных функций & » & 1 \\

    \hline
    К0403 & Отсутствие противоречий в выполнении алгоритмов & » & 1 \\

    \hline
    К0404 & Правильность взаимосвязей & » & 1 \\

    \hline
    К0405 & Правильность реализации интерфейса между модулями & » & 1 \\

    \hline
    К0406 & Правильность реализации интерфейса с пользователем & » & 1 \\

    \hline
    К0407 & Отсутствие противоречий в настройке системы & » & 1 \\

    \hline
    К0408 & Отсутствие противоречий в диагностике системы & » & 1 \\

    \hline
    К0409 & Отсутствие противоречий в общих переменных & » & 1 \\

    \hline
    &&& 1 \\



    \hline
    \multicolumn{4}{|c|}{ \textbf{Единообразие интерфейсов между модулями и пользователями}} \\

    \hline
    К0501 & Единообразие способов вызова модулей & » & 1 \\

    \hline
    К0502 & Единообразие процедур возврата управления из модулей & » & 1 \\

    \hline
    К0503 & Единообразие способов сохранения информации для возврата & » & 0 \\

    \hline
    К0504  & Единообразие способов восстановления информации для возврата & » & 0 \\

    \hline
    К0505  & Единообразие организации списков передаваемых параметров & » & 0 \\

    \hline
    &&& 0,4 \\



    \hline
    \multicolumn{4}{|c|}{ \textbf{Единообразие кодирования и определения переменных}} \\

    \hline
    К0601 & Единообразие наименования каждой переменной и константы & » & 1 \\

    \hline
    К0602 & Все ли одинаковые константы встречаются во всех программах под одинаковыми именами & » & 0 \\

    \hline
    К0603 & Единообразие определения внешних данных во всех программах & » & 1 \\

    \hline
    К0604 & Используются ли разные идентификаторы для разных переменных & » & 1 \\

    \hline
    К0605 & Все ли общие переменные объявлены как общие переменные & » & 1 \\

    \hline
    К0606 & Наличие определений одинаковых атрибутов & » & 1 \\

    \hline
    &&& 0,8 \\



    \hline
    \multicolumn{4}{|c|}{ \textbf{Соответствие документации стандартам}} \\

    \hline
    К0701 & Комплектность документации в соответствии со стандартами & » & 1 \\

    \hline
    К0702 & Правильное оформление частей документов & » & 1 \\

    \hline
    К0703 & Правильное оформление титульных и заглавных листов документов & » & 1 \\

    \hline
    К0704 & Наличие в документах всех разделов в соответствии со стандартами & » & 1 \\

    \hline
    К0705 & Полнота содержания разделов в соответствии со стандартами & » & 0 \\

    \hline
    К0706 & Деление документов на структурные элементы: разделы, подразделы, пункты, подпункты & » & 1 \\

    \hline
    &&& 0,8 \\



    \hline
    \multicolumn{4}{|c|}{ \textbf{Соответствие ПС стандартам программирования}} \\

    \hline
    К0801 & Соответствие организации и вычислительного процесса эксплуатационной документации & » & 1 \\

    \hline
    К0802 & Правильность заданий на выполнение программы, правильность написания управляющих и операторов (отсутствие ошибок) & » & 1 \\

    \hline
    К0803& Отсутствие ошибок в описании действий пользователя & » & 1 \\

    \hline
    К0804 & Отсутствие ошибок в описании запуска & » & 1 \\

    \hline
    К0805 & Отсутствие ошибок в описании генерации & » & 1 \\

    \hline
    К0806 & Отсутствие ошибок в описании настройки & » & 1 \\

    \hline
    &&& 1 \\



    \hline
    \multicolumn{4}{|c|}{ \textbf{Полнота тестирования проекта}} \\

    \hline
    К1001 & Наличие требований к тестированию программ & » & 0 \\

    \hline
    К1002 & Достаточность требований к тестированию программ & » & 0 \\

    \hline
    К1003 & Отношение числа модулей, отработавших в процессе тестирования и отладки (Qтм) к общему числу модулей (Qтм) & Расчетный & 1 \\

    \hline
    К1004 & Отношение числа логических блоков, отработавших в процессе тестирования и отладки (Qтб), к общему числу логических блоков в программе (Qтб) & То же & 1 \\

    \hline
    &&& 0,7 \\



    \hline
\end{ztable}


Абсолютные показатели критериев i-ого фактора качества определяется по формуле:
$$P_i = \sum_{k=0}^{n}(P^M_{jk} * V^M_{jk})$$
, где

$P^M_{jk}$ - итоговая оценка k-той метрики j-того критерия;

$V^M_{jk}$ - весовой коэффициент j-того показателя;

$n$ - число метрик, относящихся к j-тому критерию.

Таким образом, абсолютные показатели составляют:

\begin{ztable}{|p{5cm}|p{5cm}|}{ Результаты оценки качества программного продукта} {quality_result_table}
  \hline
  Фактор качества & Оценка\\

  \endhead

  \hline
  Надёжность & 0,7 \\
  \hline
  Сопровождаемость & 0,5 \\
  \hline
  Удобство применения & 0,9 \\
  \hline
  Эффективность & 0,9 \\
  \hline
  Универсальность & 0,7 \\
  \hline
  Корректность & 0,7 \\
  \hline
\end{ztable}

Все показатели принимают значения в пределах требуемой нормы.

\textbf{Выводы}

В результате проделанной работы была произведена оценка качества программного продукта
“Программа для расчета и предупреждения пересечения стволов нефтяных скважин”.

Показатель оценки надёжности равен 0,7. Эта величина показывает, что программа оснащена определенными базовыми
методами защиты от сбоев и злоумышленников.

Высокий показатель универсальности равен 0,7 говорит о том, что данный модуль может быть перенесен в другие приложения.

Значение показателя сопровождаемости равное 0,5 говорит о необходимости дальнейшей работы по улучшению наглядности и
устойчивости функционирования.

Полученные оценки 0,9 означают, что программа достаточно эффективна и удобна в применении.
