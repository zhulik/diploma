\textbf{Метрическая оценка качества программного продукта.}

В данной части дипломной работы проводится оценка качества программного продукта согласно ГОСТ 28195-89.

\textbf{Определение подкласса программных средств}

Данное программное средство относится к подклассу 509 – Прочие ПС.

\textbf{Показатели надежнос­ти программного средства}

Оценочные элементы фактора “Надежность ПС”

\begin{ztable}{Оценочные элементы фактора “Надежность ПС”}{table1}
  \begin{tabularx}{\textwidth}{|l|X|c|l|}
    \hline
    Код элемента & Наименование & Метод оценки & Оценка\\

    \hline
    \multicolumn{4}{|c|}{\textbf{Средства восстановления при ошибках на входе}} \\

    \hline
    Н0101 & Наличие требований к программе по устойчивости функционирования при на­личии ошибок во входных данных  & Экспертный & 1 \\

    \hline
    Н0102 & Возможность обработки ошибочных ситуаций & То же & 1 \\

    \hline
    Н0103 & Полнота обработки оши­бочных ситуаций & » & 1 \\

    \hline
    Н0104 & Наличие тестов для про­верки допустимых значений входных данных & » & 0 \\

    \hline
    Н0105 & Наличие системы контро­ля полноты входных дан­ных & » & 0 \\

    \hline
    Н0106 & Наличие средств контро­ля корректности входных данных & » & 1 \\

    \hline
    Н0107 & Наличие средств контро­ля непротиворечивости входных данных & » & 0 \\

    \hline
    Н0108 & Наличие проверки пара­метров и адресов по диа­пазону их значений & » & 1 \\
    Н0109 & Наличие обработки гра­ничных результатов & » & 1 \\

    \hline
    Н0110 & Наличие обработки неоп­ределенностей & » & 0,6 \\

    \hline
    &&& 0,8 \\

    \hline
    \multicolumn{4}{|c|}{\textbf{Средства восстановления при сбоях оборудования}} \\

    \hline
    Н0201 & Наличие требований к программе по восстановле­нию процесса выполнения в случае сбоя операцион­ной системы, процессора, внешних устройств & » & 0 \\

    \hline
    Н0202 & Наличие требований к программе по восстановле­нию результатов при отка­зах процессора, ОС & » & 1 \\

    \hline
    Н0203 & Наличие средств восста­новления процесса в слу­чае сбоев оборудования & » & 0 \\

    \hline
    Н0204 & Наличие возможности разделения по времени вы­полнения отдельных функ­ций программ & » & 1 \\

    \hline
    Н0205 & Наличие возможности повторного старта с точки останова & » & 1 \\

    \hline
    &&& 0,6 \\

    \hline
    \multicolumn{4}{|c|}{\textbf{Реализация управления средствами восстановления}} \\

    \hline
    Н0301  & Наличие централизован­ного управления процесса­ми, конкурирующими из-за ресурсов & » & 1 \\

    \hline
    Н0302  & Наличие возможности ав­томатически обходить оши­бочные ситуации в процессе вычисления & » & 0 \\

    \hline
    Н0303 & Наличие средств, обеспечивающих завершение про­цесса решения в случае по­мех & » & 1 \\

    \hline
    Н0304  & Наличие средств, обеспе­чивающих выполнение про­граммы в сокращенном объеме в случае ошибок или помех & » & 0 \\

    \hline
    Н0305 & Показатель устойчивости к искажаемым воздействи­ям& Расчетный & 0 \\

    \hline
    &&& 0,4 \\



    \hline
  \end{tabularx}
\end{ztable}
