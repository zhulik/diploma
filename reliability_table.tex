\textbf{Показатели надёжности программного средства}

\begin{ztable}{|p{2cm}|p{10cm}|p{2cm}|p{2cm}|}{Оценочные элементы фактора “Надёжность ПС”}{reliability_table}
    \hline
    Код элемента & Наименование & Метод оценки & Оценка\\

    \endhead

    \hline
    \multicolumn{4}{|c|}{\textbf{Средства восстановления при ошибках на входе}} \\

    \hline
    Н0101 & Наличие требований к программе по устойчивости функционирования при наличии ошибок во входных данных  & Экспертный & 1 \\

    \hline
    Н0102 & Возможность обработки ошибочных ситуаций & То же & 1 \\

    \hline
    Н0103 & Полнота обработки ошибочных ситуаций & » & 1 \\

    \hline
    Н0104 & Наличие тестов для проверки допустимых значений входных данных & » с& 0 \\

    \hline
    Н0105 & Наличие системы контроля полноты входных данных & » & 0 \\

    \hline
    Н0106 & Наличие средств контроля корректности входных данных & » & 1 \\

    \hline
    Н0107 & Наличие средств контроля непротиворечивости входных данных & » & 0 \\

    \hline
    Н0108 & Наличие проверки параметров и адресов по диапазону их значений & » & 1 \\
    Н0109 & Наличие обработки граничных результатов & » & 1 \\

    \hline
    Н0110 & Наличие обработки неопределенностей & » & 0,6 \\

    \hline
    &&& 0,8 \\

    \hline
    \multicolumn{4}{|c|}{\textbf{Средства восстановления при сбоях оборудования}} \\

    \hline
    Н0201 & Наличие требований к программе по восстановлению процесса выполнения в случае сбоя операционной системы, процессора, внешних устройств & » & 0 \\

    \hline
    Н0202 & Наличие требований к программе по восстановлению результатов при отказах процессора, ОС & » & 1 \\

    \hline
    Н0203 & Наличие средств восстановления процесса в случае сбоев оборудования & » & 0 \\

    \hline
    Н0204 & Наличие возможности разделения по времени выполнения отдельных функций программ & » & 1 \\

    \hline
    Н0205 & Наличие возможности повторного старта с точки останова & » & 1 \\

    \hline
    &&& 0,6 \\

    \hline
    \multicolumn{4}{|c|}{\textbf{Реализация управления средствами восстановления}} \\

    \hline
    Н0301  & Наличие централизованного управления процессами, конкурирующими из-за ресурсов & » & 1 \\

    \hline
    Н0302  & Наличие возможности автоматически обходить ошибочные ситуации в процессе вычисления & » & 0 \\

    \hline
    Н0303 & Наличие средств, обеспечивающих завершение процесса решения в случае помех & » & 1 \\

    \hline
    Н0304  & Наличие средств, обеспечивающих выполнение программы в сокращенном объеме в случае ошибок или помех & » & 0 \\

    \hline
    Н0305 & Показатель устойчивости к искажаемым воздействиям& Расчетный & 0 \\

    \hline
    &&& 0,4 \\

    \hline
    \multicolumn{4}{|c|}{\textbf{Функционирование в заданных режимах}} \\

    \hline
    Н0401 & Вероятность безотказной работы & То же & 1 \\

    \hline
    \multicolumn{4}{|c|}{\textbf{Обеспечение обработки заданного объема информации}} \\

    \hline
    Н0501 & Оценка по среднему времени восстановления & » & 1 \\

    \hline
    Н0502 & Оценка по продолжительности преобразования входного набора данных в выходной & » & 1 \\

    \hline
    &&& 1 \\
    \hline
\end{ztable}
