\textbf{Наименование программы}

"Приложение для предупреждения пересечения стволов скважин "Collisions"

\textbf{Характеристика области применения программы}

Программа (“Collisions”) разрабатывается в рамках выпускной квалификационной работы. Программа находится в стадии внедрения.

\textbf{Основания для разработки}

Разработка программного обеспечения ведется в соответствии с заданием на дипломное проектирование, составленным совместно с
руководителем дипломной работы и утвержденным кафедрой ВМиК.

\textbf{Назначение разработки}

Программа предназначена для визуализации, усреднения замеров стволов скважин, а так же для предупреждения их пересечения.

\textbf{Требования к функциональным характеристикам}

Данный программный комплекс должен обладать следующими функциями:
\begin{itemize}
  \item внесение, редактирование и удаление данных о подрядчиках;
  \item внесение, редактирование и удаление данных о клиентах;
  \item внесение, редактирование и удаление данных о месторождениях;
  \item внесение, редактирование и удаление данных о кустах;
  \item внесение, редактирование и удаление данных о скважинах;
  \item внесение, редактирование и удаление данных о стволах;
  \item внесение, редактирование и удаление данных о замерах;
  \item импорт данных замера из csv файлов, полученных от измерительного оборудования;
  \item визуализация кустов, скважин, стволов и замеров;
  \item усреднение замеров ствола;
  \item расчет и визуализация расстояний между стволами скважин в кусте;
\end{itemize}

\textbf{Требования к надёжности}

Программный продукт (ПП) должен обеспечивать: устойчивую и корректную работу с базой данных,
сохранность информации в случаях возникновения сбоев.

\textbf{Требования к обеспечению надёжного функционирования программы}

Надёжное (устойчивое) функционирование программы должно быть обеспечено выполнением
Заказчиком совокупности организационно-технических мероприятий, перечень которых приведен ниже:

а) организацией бесперебойного питания технических средств;

б) регулярным выполнением рекомендаций Министерства труда и социального развития РФ,
изложенных в Постановлении от 23 июля 1998 г. Об утверждении межотраслевых типовых норм
времени на работы по сервисному обслуживанию ПЭВМ и оргтехники и сопровождению программных средств»;

в) регулярным выполнением требований ГОСТ 51188-98. Защита информации. Испытания программных средств на наличие компьютерных вирусов

\textbf{Время восстановления после отказа}

Время восстановления после отказа, вызванного сбоем электропитания технических средств (иными внешними факторами),
не фатальным сбоем (не крахом) операционной системы, не должно превышать 30-ти минут при условии соблюдения условий
эксплуатации технических и программных средств.

Время восстановления после отказа, вызванного неисправностью технических средств, фатальным сбоем (крахом) операционной
системы, не должно превышать времени, требуемого на устранение неисправностей технических средств и переустановки программных средств.

\textbf{Отказы из-за некорректных действий пользователей системы}

Отказы программы вследствие некорректных действий пользователя при взаимодействии с программой недопустимы.

\textbf{Требования к квалификации и численности персонала}

Минимальное количество персонала, требуемого для работы программы, должно составлять не менее 2 штатных единиц —
системный администратор и конечный пользователь программы — оператор.

В перечень задач, выполняемых системным администратором, должны входить:

а) задача поддержания работоспособности технических средств;

б) задачи установки (инсталляции) и поддержания работоспособности системных программных средств — операционной системы;

в) задача установки (инсталляции) программы.

\textbf{Требования к составу и параметрам технических средств}

Для выполнения программы желательна следующая аппаратная конфигурация:

\begin{itemize}
  \item ПК с x86 - совместимым процессором 1ГГц и выше;
  \item оперативная память - не менее 512Мб;
  \item минимум 200Мб свободного пространство на диске;
  \item OS Windows 7 или старше или ОС на базе ядра Linux
\end{itemize}

\textbf{Требования к организации входных данных}

Входными данными являются:
\begin{itemize}
  \item данные замера ствола, полученные от измерительного оборудования;
  \item данные о подрядчиках, клиентах, месторождениях, кустах,скважинах, стволах и замерах;
\end{itemize}

\textbf{Требования к формированию выходных данных}

Выходными данными являются графическое представление замеров стволов, а так же графическое представление расстояний
между стволами.

\textbf{Требования к реализуемым методам решения}

Методы решения, используемые в работе программы, должны быть эффективными и высокопроизводительными,
позволять получать верный результат за приемлемое время, а также контролировать случаи возникновения некорректной работы.

\textbf{Требования к исходным кодам и языкам программирования}

Система должна быть написана на языке C++ и иметь удобный графический интерфейс.

\textbf{Состав и требования к программной документации}

В состав программной документации должны входить:
\begin{itemize}
  \item техническое задание;
  \item руководство программиста;
  \item руководство пользователя.
\end{itemize}
