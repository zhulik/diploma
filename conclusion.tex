\newpage
\section*{ЗАКЛЮЧЕНИЕ}
\addcontentsline{toc} {section} {ЗАКЛЮЧЕНИЕ}

В представленной работе описан программный продукт для предотвращения пересечения стволов скважин.
В процессе разработки были достигнуты следующие цели:

\begin{itemize}
  \item проведен анализ процесса кустового бурения скважин и проблемы пересечения стволов;
  \item проведен анализ известного программного обеспечения для анализа данных инклинометрии и оценки расстояния между стволами;
  \item разработаны функциональная и информационная модели программного обеспечения;
  \item разработано алгоритмическое и программное обеспечения для визуализации, усреднения и анализа замеров стволов скважин. Были разработаны модули для
    импорта, редактирования, визуализации и усреднения замеров стволов, а так же модуль для оценки и визуализации расстояний между ними. 
  \item произведено тестирование и анализ эффективности разработанного ПО.
\end{itemize}

В процессе тестирования продукт показал свою корректность в сравнении с результатами ручного расчета. Анализ эффективности показал
снижение затрат на владение IT-инфраструктурой. Результаты внедрены на ООО НПП "Сириус"

Такой программный продукт может найти применение в строительстве с использованием технологии искусственного замораживания
грунтов, а так же в бурении нефтяных и газовых скважин. Он способен успешно заменить часть функционала таких дорогих и сложных
программных продуктов, как Landmark Compass и Paradigm Sysdrill, таким образом может помочь пользователю серьезно сэкономить.

Для работы с данной программой требуется знания в области нефтяной промышленности, а также понимание принципов инклинометрии.
Требования к наличию знаний компьютерных навыков минимальна.
