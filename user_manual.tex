\textbf{Назначение и условия применения программы}

Приложение для рассчета вероятности пересечения стволов нефтяных скважин.
Пользование программой не требует специальной квалифицированной  подготовки.

\textbf{Условия применения программы}

Программный продукт должен работать на любых ПК с x86-совместимым процессором с частотой 1Ггц и выше,
оперативной памятью не менее 512мб и досупным дисковым пространством минимум 200Мб, работающий под
управлением ОС Windows 7 или ОС на базе ядра Linux.

\textbf{Требования к квалификации пользователя программы}

\begin{itemize}
  \item{знакомство с любой из поддерживаемых ОС}
  \item{знакомство с руководством пользователя}
  \item{знакомство с руководством пользователя}
\end{itemize}

\textbf{Установка программы}

Копировать директорию с ПП на компьютер, при необходимости создать на рабочем столе(зависит от ОС)

\textbf{Запуск программы}

Для запуска программы необходимо исполнить бинарный файл Collisions(или Collisions.exe для ОС Windows)

\textbf{Интерфейс программы}

После запуска программы открывается главное окно программы(\fullref{manual/1.jpg})

\putimage{manual/1.jpg}{
  Главное окно программы: (1) главное меню программы (2) панель инструментов (3) панель дерева базы данных
}

Для добавления подрядчика в базу необходимо воспользоваться пунктом меню Program->Add contractor
(Программа->Добавить подрядчика)(\fullref{manual/2.jpg})

\putimage{manual/2.jpg}{Меню добавления нового заказчика}

В появившемся диалоге ввести название/имя подрядчика, и, при необходимости, комментарий(\fullref{manual/3.jpg})

\putimage{manual/3.jpg}{Диалог добавления нового заказчика}

После нажания Ok Подрядчик будет добавлен в базу(\fullref{manual/4.jpg})

\putimage{manual/4.jpg}{Созданный подрядчик}

Другие элементы добавляются в базу отличным от подрядчика методом. Для добавления элемента необходимо кликнуть по его родителю
(для заказчика это подрядчик, для месторождения - заказчик и т.д) в дереве базы и в открывшейся справа таблице нажать Insert.
В таблице появится пустая строка для добавления нового элемента, первое поле строки будет активно для редактирования
(\fullref{manual/5.jpg} и \fullref{manual/6.jpg})

\putimage{manual/5.jpg}{Таблица для редактирования и просмотра содержимого элемента базы}
\putimage{manual/6.jpg}{Добавление нового элемента в таблицу}

Для перемещения между полями новой строки необходимо использовать Tab, для сохранения элемента в базе - Enter. Все остальные элементы
(Месторождения, кусты, скважины, стволы, замеры и точки замера) добавляются аналогично. См. изображения
(\fullref{manual/7.jpg}, \fullref{manual/8.jpg} и \fullref{manual/6.jpg})
\putimage{manual/7.jpg}{Таблица месторождений}
\putimage{manual/8.jpg}{Добавление месторождения}
\putimage{manual/9.jpg}{Таблица стволов}

Программа поддерживает импорт замеров из буфера обмена и некоторых форматов текстовых файлов(например, csv).
Для импортирования замера необходимо выбрать ствол в дереве базы и нажать на кнопку Import(Импорт) на панели инструментов.
(\fullref{manual/import_button.jpg})

\putimage{manual/import_button.jpg}{Кнопка импорта}

После нажатия появится мастер импорта замера(\fullref{manual/12.jpg})

\putimage{manual/12.jpg}{
  Мастер импорта замера: (1,2) источник импорта, файл или буфер обмена (3) название замера (4) дата создания замера (4) символ, используемых
    для разделения столбцов(для CSV это обычно запятая или точка с запятой, для буфера обмена - символ табуляции)
}

При импорте из файла необходимо ввести путь к файлу в соответствующее поле или нажать на кнопку "...".
При нажатии на нее откроется стандартный диалог выбора файлов. После выбора файла(если импорт проиходит не из буфера обмена) и
заполнения остальных полей формы можно перейти на следующую страницу мастера, это делается кнопкой Next(Далее).
На следущей странице необходимо выбрать те столбцы и строки, которых содержат необходимы данные. Для импотра замеров необходимы: измеренная глубина, зенит и азимут.
Выделить данные в таблице можно зажав правую кнопку мыши и потянув курсов в нужную сторону.
(\fullref{manual/14.jpg} и \fullref{manual/15.jpg})

\putimage{manual/14.jpg}{Страница с данными}
\putimage{manual/15.jpg}{Выбор элементов с данными}

После выделения элементов с данными можно переходить на следующую страницу мастера.(\fullref{manual/16.jpg})

\putimage{manual/16.jpg}{Страница выбора столбцов с данными}

На этой странице необходимо выбрать какие столбцы содержат нужные данные(глубину, зенит и азимут). В выпадающих списках вверху диалога нужно выбрать соответствующие номера столбцов.
После нажатия кнопки Finish(Завершить) в базу будут добавлен замер и соответвующие точки (\fullref{manual/17.jpg})

\putimage{manual/17.jpg}{Импортированный замер}

Для создания усредненного замера ствола нужно воспользоваться кнопкой Create average(Создать усредненный) на панели инструментов

\putimage{manual/average_button.jpg}{Кнопка добавления усредненного замера}

После нажания в базу будет добавлен усреденный замер с именем <Имя ствола>-average. Если замер у ствола был один, то он продублируется.

Проектные замеры необходимы, чтобы для каждого ствола можно было индивидуально задать минимально и максимально допустимые расстояния.
В случае, если проектный замер для ствола задан, будут использоваться его настройки, иначе - указанные в настройках.

Приложение Collisions позволяет визуализировать отдельные замеры, скважины и кусты, для отображение графика элемента нужно использовать кноку Plot(График) на панели инструментов
(\fullref{manual/plot_button.jpg})

\putimage{manual/plot_button.jpg}{Кнопка просмотра графика}

После нажатия кнопки появится диалог просмотра (\fullref{manual/18.jpg})

\putimage{manual/18.jpg}{
    Даилог просмотра графиков: (1) просмотр плана (2) просмотр профиля (3) управление отображеним графика(показ/скрытие осей и сетки) (4) автоматическа раскраска графиков
    (5) - легенда, настроки цвета и отображение графиков.
}

В открывшемся диалоге будут изображены:
\begin{itemize}
  \item Для замера - его графики
  \item Для ствола - все его замеры
  \item Для скважины - усредненные замеры всех ее стволов
  \item Для куста - все его скважины
\end{itemize}

Диалог настроек позволяет настроить значения по-умолчанию для минимально и максимально допустимого расстояния между стволами.
(\fullref{manual/settings.jpg})

\putimage{manual/settings.jpg}{Диалог настроек}

Для просмотра графика расстояний между стволами нужно выбрать куст в дереве базы и воспользоваться кнопкой Anticollision(Предупреждение пересечения стволов) на панели иструментов.
В открывшемся диалоге необходимо выбрать ствол, который будет принят за основной, а так же указать с какими стволами будут расчитываться расстояния

Для просмотра графика расстояний между стволами нужно выбрать куст в дереве базы и воспользоваться кнопкой Anticollision(Предупреждение пересечения стволов) на панели иструментов.
В открывшемся диалоге необходимо выбрать ствол, который будет принят за основной, а так же указать с какими стволами будут расчитываться расстояния
(\fullref{manual/32.jpg})

\putimage{manual/32.jpg}{
Диалог настроек процесса предупреждение пересечения стволов: (10) выбор базовой скважины (2) выбор базового ствола (3) дерево стволов для отображениия}

После нажатия Ok откроется диалог с графиком расстояний. Зеленой зоной выделено расстояние между минимально и максимально допустимыми расстояними.
(\fullref{manual/anticollision.jpg})

\putimage{manual/anticollision.jpg}{Диалог графика предупреждения пересечения стволов}

Для каждого ствола скважины можно задать координаты устья либо выбрать точку из соседних стволов этой же скважины. Окно свойств вызывется из контекстного меню ствола в дереве базы данных
(\fullref{manual/wellboreprops0.png}, \fullref{manual/wellboreprops1.png} и \fullref{manual/wellboreprops2.png})

\putimage{manual/wellboreprops0.png}{Меню свойств ствола}
\putimage{manual/wellboreprops1.png}{Установка устья ствола вручную}
\putimage{manual/wellboreprops2.png}{Установка устья ствола из соседнего ствола той же скважины}
