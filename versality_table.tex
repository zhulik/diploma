\textbf{Показатели универсальности}

\begin{ztable}{|p{2cm}|p{10cm}|p{2cm}|p{2cm}|}{ Оценочные элементы фактора “Универсальность ПС”}{versality_table}
    \hline
    Код элемента & Наименование & Метод оценки & Оценка\\

    \endhead

    \hline
    \multicolumn{4}{|c|}{ \textbf{Зависимость от используемого комплекса технических средств}} \\

    \hline
    Г0701 & Оценка зависимости программ от ёмкости оперативной памяти ЭВМ  & » & 1 \\

    \hline
    Г0702 & Оценка зависимости временных характеристик программы от скорости вычислений ЭВМ  & » & 1 \\

    \hline
    Г0703 & Оценка зависимости функционирования программы от числа внешних запоминающих устройств и их общей емкости  & » & 0 \\

    \hline
    Г0704 & Оценка зависимости функционирования программы от специальных устройств ввода-вывода & » & 1 \\

    \hline
    &&& 0,7 \\

    \hline
    \multicolumn{4}{|c|}{ \textbf{Зависимость от базового программного обеспечения}} \\

    \hline
    Г0801 & Применение специальных языков программирования & » & 1 \\

    \hline
    Г0802 & Оценка зависимости программы от программ  операционной системы  & » & 1 \\

    \hline
    Г0803 & Зависимость от других программных средств & » & 1 \\

    \hline
    &&& 1 \\

    \hline
    \multicolumn{4}{|c|}{ \textbf{Изоляция немобильности}} \\

    \hline
    Г0901 & Оценка локализации непереносимой части программы & » & 1 \\

    \hline
    &&& 1 \\

    \hline
    \multicolumn{4}{|c|}{ \textbf{Простота кодирования}} \\

    \hline
    Г1001& Оценка использования отрицательных или булевых выражений & » & 1 \\

    \hline
    Г1002 & Оценка программы по использованию условных переходов & » & 1 \\

    \hline
    Г1003& Оценка программы по использованию безусловных переходов  & » & 0 \\

    \hline
    Г1004 & Оформление процедур входа и выхода из циклов  & » & 1 \\

    \hline
    Г1005 & Ограничения на модификацию переменной индексации в цикле  & » & 1 \\

    \hline
    Г1007 & Оценка программы по использованию локальных переменных  & » & 1 \\

    \hline
    Г1006 & Оценка модулей по направлению потока управления & » & 0 \\

    \hline
    &&& 0,7 \\

    \hline
    \multicolumn{4}{|c|}{ \textbf{Число комментариев}} \\

    \hline
    Г1101 & Оценка программы по числу комментариев & » & 1 \\

    \hline
    &&& 1 \\

    \hline
    \multicolumn{4}{|c|}{ \textbf{Качество комментариев}} \\

    \hline
    Г1201 & Наличие заголовка в программе  & » & 1 \\

    \hline
    Г1202& Комментарии к точкам ветвлений  & » & 1 \\

    \hline
    Г1203 & Комментарии к машинозависимым частям программы & » & 1 \\

    \hline
    Г1204 & Комментарии к машинозависимым операторам программы  & » & 0 \\

    \hline
    Г1205 & Комментарии к операторам объявления переменных  & » & 1 \\

    \hline
    Г1206 & Оценка семантики операторов  & » & 1 \\

    \hline
    Г1207 & Наличие соглашений по форме представления комментариев  & » & 0 \\

    \hline
    Г1208 & Наличие общих комментариев к программам & » & 1 \\

    \hline
    &&& 0,7 \\

    \hline
    \multicolumn{4}{|c|}{ \textbf{Использование описательных средств языка}} \\

    \hline
    Г1301 & Использование языков высокого уровня & » & 1 \\

    \hline
    Г1302& Семантика имен используемых переменных& » & 1 \\

    \hline
    Г1303 & Использование отступов, сдвигов и пропусков при формировании текста & » & 1 \\

    \hline
    Г1304& Размещение операторов по строкам & » & 1 \\

    \hline
    &&& 1 \\

    \hline
    \multicolumn{4}{|c|}{ \textbf{Независимость модулей}} \\


    \hline
    Г1401 & Передача информации для управления по параметрам & » & 1 \\

    \hline
    Г1402 & Наличие передачи результатов работы между модулями & » & 1 \\

    \hline
    Г1403 & Наличие проверки правильности данных, получаемых модулями от вызываемого модуля & » & 1 \\

    \hline
    Г1404 & Использование общих областей памяти & » & 1 \\

    \hline
    Г1405 & Параметрическая передача входных данных & » & 1 \\


    \hline
    &&& 1 \\


    \hline
\end{ztable}
