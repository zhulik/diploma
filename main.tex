\documentclass[12pt]{article}
\usepackage{fontspec}
\usepackage{array}
\usepackage{indentfirst}
\usepackage{fancyhdr}
\usepackage{ragged2e}
\usepackage{graphicx}
\usepackage{float}
\usepackage{caption}
\usepackage{tabularx}
\usepackage{longtable}



\graphicspath{{images/}}

\newcommand{\putimage}[2] {
  \begin{figure}[H]
  \center{\includegraphics[width=1\linewidth]{#1}}
  \caption{#2}
  \label{ris:#1}
  \end{figure}
}

\newcommand{\fullref}[1] {
  \figurename \ref{ris:#1}
}

\newenvironment{ztable}[3]
{\begin{longtable}[H]{#1}
  \captionsetup{justification=raggedleft,singlelinecheck=false}
  \caption{#2 \label{tab:#3}} \\
}
{\end{longtable}}


\setmainfont[Mapping=tex-text]{Times New Roman}

\pagestyle{fancy}
\fancyhf{}
\fancyhead[R]{\thepage}
\renewcommand{\headrulewidth}{0pt}
\fancyheadoffset{0cm}

\renewcommand{\figurename}{Рис.}
\renewcommand{\tablename}{Таблица }
\renewcommand{\contentsname}{Оглавление}
\renewcommand\thesection{\arabic{section}.}
\renewcommand\thesubsection{\arabic{section}.\arabic{subsection}}

\usepackage{geometry} % Меняем поля страницы
\geometry{left=3cm}% левое поле
\geometry{right=1cm}% правое поле
\geometry{top=2cm}% верхнее поле
\geometry{bottom=2cm}% нижнее поле

\sloppy
\begin{document}
\begin{titlepage}
\newpage

\begin{center}
Федеральное государственное бюджетное образовательное учреждениe\\
высшего профессионального образования \\
\vspace{0.5cm}
\footnotesize УФИМСКИЙ ГОСУДАРСТВЕННЫЙ АВИАЦИОННЫЙ ТЕХНИЧЕСКИЙ УНИВЕРСИТЕТ\\
\vspace{0.5cm}
ФАКУЛЬТЕТ  ИНФОРМАТИКИ  И  РОБОТОТЕХНИКИ \\
\vspace{0.5cm}
КАФЕДРА  ВЫЧИСЛИТЕЛЬНОЙ  МАТЕМАТИКИ  И  КИБЕРНЕТИКИ\\
\vspace{0.5cm}
\normalsize Направление 231000 – Программная инженерия
\end{center}
 
\vspace{2em}

\begin{center}
\textbf{ВЫПУСКНАЯ КВАЛИФИКАЦИОННАЯ РАБОТА}
\end{center}

\vspace{2.5em}
 
\begin{center}
\textsc{Тема: Программа для рассчета вероятности пересечения стволов нефтяных скважин}
\end{center}

\vspace{3em}
 
\begin{table}[H]
\begin{tabularx}{\textwidth}{|l|c|l|l|}
\hline
& ФИО & Подпись & Дата\\
\hline
Студент & Синявский Г. Н. &&\\
\hline
Руководитель работы & Еникеева К. Р. &&\\
\hline
Консультант & Еникеева К. Р. &&\\
\hline
Контроль программного продукта &&&\\
\hline
Председатель комиссии по предзащите &&&\\
\hline
Рецензент &&& \\
\hline

\end{tabularx}
\end{table} 

\footnotesize
\vspace{1em}
\center Допущен к защите\\
Зав. кафедрой  ВМК, д.т.н., проф.\\
\underline{\hspace{5cm}} Н.И. Юсупова\\
“\underline{\hspace{1cm}}”\underline{\hspace{5cm}} 2015 г.
\normalsize
\vspace{\fill}

\begin{center}
Уфа~--- 2015г.
\end{center}

\end{titlepage}
\begin{titlepage}
\newpage

\begin{center}
Федеральное государственное бюджетное образовательное учреждениe\\
высшего профессионального образования \\
\vspace{0.5cm}
УФИМСКИЙ ГОСУДАРСТВЕННЫЙ АВИАЦИОННЫЙ ТЕХНИЧЕСКИЙ УНИВЕРСИТЕТ\\
\vspace{0.5cm}
ФАКУЛЬТЕТ  ИНФОРМАТИКИ  И  РОБОТОТЕХНИКИ \\
\vspace{0.5cm}
КАФЕДРА  ВЫЧИСЛИТЕЛЬНОЙ  МАТЕМАТИКИ  И  КИБЕРНЕТИКИ\\
\vspace{0.5cm}
Направление 231000 – Программная инженерия
\end{center}
\vspace{0.5cm}
\begin{flushright}
"УТВЕРЖДАЮ"
Зав. кафедрой  ВМК, д.т.н., проф.\\
\vspace{0.5cm}
\underline{\hspace{5cm}} Н.И. Юсупова\\
\underline{\hspace{1cm}}”\underline{\hspace{5cm}} 2015 г.
\end{flushright}
\vspace{1cm}

\begin{center}
  \Large{ ЗАДАНИЕ }
\vspace{0.5cm}

на подготовку выпускной квалификационной работы
\end{center}

студента Синявского Глеба Николаевича

\begin{enumerate}
  \item Тема работы - Программа для рассчета вероятности пересечения стволов нефтяных скважин\\
  ( утверждена распоряжением по факультету No 100500 от “01” Июня 2015г. )
  \item Срок представления работы “01” Января 2015г.
  \item Описание задачи\\
    Необходимо разработать программный продукт, позволяющий усреднять и визуализировать замеры стволов нефтяных скважен, а так же
    позволять оценивать вероятность пересечения стволов.
  \item Математическая часть\\
  ???
  \item Спецификация входных и выходных данных\\
  Входные данные - csv-файлы, содержащие результаты замера ствола скважины. Выходные - визуализация скважины в пространстве, визуализации оценки расстояний
  между стволами.
  \item Применяемые инструментальные средства\\
  Библиотека построение графического интерфейса - Qt. СУБД - SQLite. Библиотека\\визуализации - MathGL.
  \item Особые условия эксплуатации программного продукта\\
  Основная ОС для запуска программного продукта - Windows 7 и старше, но продукт должен разрабатываться как кросс-платформенный и иметь возможность запуска
  под управлением ОС Linux.
  \item Дополнительные условия\\
  Продукт должен иметь возможность импортировать csv произвольного формата, для этого должен быть разработан мастер импорта, позволяющий
  выбирать диапазон ячеек таблицы и указывать их тип.
\end{enumerate}

\vspace{\fill}

Руководитель работы \underline{\hspace{5cm}}

Консультант \underline{\hspace{6.5cm}}
\end{titlepage}

\tableofcontents % это оглавление, которое генерируется автоматически
\newpage
\section*{ Аннотация}
\addcontentsline{toc} {section} {Аннотация}
Представленная дипломная работа содержит описание программного решения задачи предупреждения пересечения стволов скважин.

В работе было проведено исследование данной задачи, ее анализ и декомпозиция на подзадачи. В соответствии с полученной структурой была
разработана информационная и функциональные модели решения задачи.

В рамках дипломной работы было разработано и внедрено в эксплуатацию программное обеспечение, реализующее поставленную задачу.

В соответствующих разделах работы приводится описание технологической базы созданного продукта, анализируются имеющиеся аналоги, описывается структура и
процесс функционирования созданного программного обеспечения.

\newpage
\section*{Введение}
\addcontentsline{toc}{section}{Введение}
%TODO: написать введение
\subsection*{Описание предметной области}
\subsection*{Мотивация, актуальность проблемы}
\subsection*{Цели, задачи ВКР}
\subsection*{Содержание работы по главам}

\newpage
\section{Анализ проблемы и постановка задачи}
\subsection{Анализ предметной области}
Бурение скважин — это процесс сооружения направленной цилиндрической горной выработки в земле, диаметр "D" которой мал по сравнению с её длиной по стволу "H",
без доступа человека на забой. Начало скважины на поверхности земли называют устьем, дно — забоем, а стенки скважины образуют ее ствол.

Кустовое бурение — сооружение скважин (в основном наклонно направленных), устья которых группируются на близком расстоянии друг от друга на общей ограниченной
площадке (основании). Применяется при разработке месторождений под застроенными участками, при разработке нефтяных и газовых месторождений в определённых
климатических условиях (например, в зимний период, когда наблюдается большой снеговой покров, или весной во время распутицы и значительных паводков),
на территории с сильно пересечённым рельефом местности или в пределах акваторий.

В настоящее время практически все эксплуатационные скважины бурятся кустовым методом, когда устья нескольких скважин в кусте расположены близко друг к другу (4–5
м) на одной технологической площадке, а забои находятся в узлах сетки разработки. Число скважин в кусте колеблется от  2 до нескольких десятков.

Искусственное искривление скважин применяется с целью:
\begin{itemize}
  \item добычи нефти и газа из труднодоступных участков, занятых на поверхности промышленными и жилыми объектами, оврагами, горами, реками, озерами, болотами, лесами, морями;
  \item экономии отводимых под строительство скважин плодородных земельных участков, лесов и др.;
  \item экономии затрат на строительство оснований, подъездных путей, линий электропередач, связи, трубопроводов;
  \item сокращения средств и времени на строительно-монтажные работы и обслуживание при эксплуатации скважин с близко расположенными устьями;
  \item обхода зон катастрофических поглощений, обвалов и аварий в стволе скважины;
  \item вскрытия продуктивных пластов, залегающих под пологим сбросом или между двумя параллельными сбросами;
  \item проходки стволов на нефтяные пласты, залегающие под соляными куполами, в связи с трудностью бурения через них (соль «плывет», срезает бурильные и обсадные колонны);
  \item бурения стволов для глушения открытых фонтанов и тушения пожаров;
  \item перебуривания части ствола скважины;
  \item вскрытия продуктивного пласта под определенным углом для увеличения поверхности дренажа и дебита скважины;
  \item многозабойного вскрытия продуктивного пласта.
\end{itemize}

Помимо искусственного искривления скважин так же имеет место быть самопроизвольное искривление, связанное с геологическими, техническими и технологическими факторами.

\textbf{Геологические факторы:}
\begin{itemize}
  \item Перемежаемость по твердости - чередование мягких и твердых горных пород;
  \item Жеоды - инородные тела в составе горное породы, отличающиеся по твердости;
\end{itemize}

\textbf{Технические факторы:}
\begin{itemize}
  \item несоосность вышки  относительно осей стола ротора  и шахтового направления;
  \item негоризонтальность стола ротора;
  \item использование искривленных труб (ведущих и бурильных) и труб, у которых резьбы нарезаны под углом;
  \item эксцентричное забуривание нижележащего участка скважины.
\end{itemize}

\textbf{Технологические факторы:}
\begin{itemize}
  \item влияние осевой нагрузки;
  \item влияние частоты вращения бурильной колонны.
\end{itemize}

Замораживание грунтов — искусственное охлаждение грунтов в естественном залегании до отрицательных температур с целью их упрочения и достижения необходимой степени водонепроницаемости.




\subsection{Содержательная постановка проблемы}

\subsection{Формальная постановка задачи}
Формальной постановке задачи соответствует контекстная диаграмма методологии IDEF0, описывающая входные и выходные данные, управляющие воздействия и механизмы,
влияющие на систему в целом, приведённая на рисунке 1.1.:
%TODO: Добавить картинку и проверить нумерацию

\subsection{Структура решения задачи, декомпозиция задачи на подзадачи}

\newpage
\section{Математическое и информационное обеспечение }

При проводке наклонной или горизонтальной скважины траектория бурения может не совпадать с проектным профилем скважины. Основная
задача технологии направленного бурения при этом заключается в оперативном расчёте величины и направления отклонения фактического профиля от
проектного и корректирование траектории бурения.

Помимо решения основной задачи по контролю и управлению траекторией бурения, расчёт пространственных координат и параметров фактического профиля необходим для:
\begin{itemize}
  \item исследования формы пространственно искривлённого ствола скважины с целью уточнения условий работы компоновок низа бурильной колонны;
  \item определения условий прохождения по стволу скважины бурильных и обсадных колонн;
  \item разработки мероприятий по предупреждению образования желобов в стенке ствола скважины;
  \item определения интервалов изнашивания обсадных колонн в процессе эксплуатации скважины.
\end{itemize}

Технология инклинометрии предусматривает измерение в каждой точке ствола скважины зенитного угла ($ \alpha $) и азимута ($ \varphi $), а также длины ствола от
устья скважины до каждой точки измерения. Задача расчёта траектории бурения состоит в том, чтобы на основании измерений рассчитать координаты
точек измерения в прямоугольной системе координат, связанной с устьем скважины, с точкой забуривания бокового ствола или с другой реперной
точкой. Другими словами, расчётным способом определить вертикальную глубину (Z) точки измерения, а также горизонтальные её смещения (X и Y) в
направлении Север-Юг с положительным направлением на Север и в направлении Восток-Запад с положительным направлением на Восток

\subsection{Математические модели подзадач}

\textbf{Метод среднего угла}

В данном методе интервал ($ L_{1-2} $) ствола скважины между соседними точками измерений представляется отрезком прямой. При этом зенитный угол
и азимут на протяжении данного интервала принимается равным средним арифметическим значениям соответствующих углов по концам интервала.
\begin{equation}
  \begin{split}
    \Delta x = L_{1-2} * sin \frac{\alpha_1 + \alpha_2}{2} * sin \frac{\varphi_1 + \varphi_2}{2};\\
    \Delta y = L_{1-2} * sin \frac{\alpha_1 + \alpha_2}{2} * cos \frac{\varphi_1 + \varphi_2}{2};\\
    \Delta z = L_{1-2} * cos \frac{\alpha_1 + \alpha_2}{2},
  \end{split}
\end{equation}

где $ \alpha_1 $ и $ \alpha_2 $ – зенитный угол в верхней и нижней точке измерения соответственно;\\
$ \varphi_1 $ и $ \varphi_2 $ – азимут в верхней и нижней точке измерения соответственно,\\
$ L_{1-2} $ - интервал ствола скважины между соседними точками измерений \cite{drilling}.

\textbf{Усреднение замеров}

Усреднение азимута и зенитного углов происходит методом нахождения простого среднего арифметического

\begin{equation}
  \begin{split}
    \alpha_{av}^d = \frac{\sum\limits_{i=1}^{n} \alpha_i^d}{n}\\
    \varphi_{av}^d = \frac{\sum\limits_{i=1}^{n} \varphi_i^d}{n}
  \end{split}
\end{equation}

где $ \alpha_{av}^d $ - усредненные зенитный угол для точки на глубине $d$;\\
$ \varphi_{av}^d $ - усредненный азимутный угол для точки на глубине $d$\\
$ n $ - количество замеров.

\textbf{Рассчет расстояний между стволами скважин}

Расстояние между точками замеров на заданной глубине вычисляется при помощи Евклидовой дистанции:

\begin{equation}
  distance^d = \sqrt{(X_1^d + X_2^d)^2 + (Y_1^d + Y_2^d)^2 + (Z_1^d + Z_2^d)^2 }
\end{equation}

где $ distance^d $ - расстояние между стволами на глубине $ d $;\\
$ X_1^d $, $ Y_1^d $ и $ Z_1^d $ - координаты точки замера первого ствола на глубине $ d $;\\
$ X_2^d $, $ Y_2^d $ и $ Z_2^d $ - координаты точки замера второго ствола на глубине $ d $.

\textbf{Вычисление координат точки с произвольной глубиной}

Измерительное оборудование работает дискретно, т.е. измеряет необходимые параметры не непрерывно, с некоторой частотой(каждые несколько метров
ствола скважины). Таким образом можно рассчитать только координаты точек замеров, предполагая, что между ними находятся отрезки прямой. Для вычисления
координат точек между точками замеров применяется уравнение прямой. Сначала находятся точки замеров, находящиеся выше и ниже искомой глубины. Затем
координаты рассчитываются по следующим формулам

\begin{equation}
  \begin{split}
    x_d = \frac{d-z_{top}}{(z_{bottom}-z_{top})}*(x_{bottom}-x_{top})+x_{top} \\
    y_d = \frac{d-z_{top}}{(z_{bottom}-z_{top})}*(y_{bottom}-y_{top})+y_{top}
  \end{split}
\end{equation}

где $ d $ - глубина;\\
$ x_d $ и $ y_d $ - координаты для глубины d; \\
$ z_{top} $ и $ y_{top} $ - координаты вышележащей точки замера;\\
$ z_{bottom} $ и $ y_{bottom} $ - координаты нижележащей точки замера.


\subsection{Информационная модель}

Информационная модель БД указана на \fullref{images/db.png}

\putimage{images/db.png}{
  Информационная модель
}

\subsection{Алгоритмы для подзадач}

\textbf{Алгоритм импорта замеров}

Алгоритм  замеров представлен на \fullref{schemes/averaging.png}

\putimage[0.4]{schemes/import.png}{
  Блок-схема алгоритма импорта замеров
}

\textbf{Алгоритм усреднения замеров}

Алгоритм усреднения замеров представлен на \fullref{schemes/averaging.png}

\putimage[0.75]{schemes/averaging.png}{
  Блок-схема алгоритма усреднения замеров
}

\textbf{Алгоритм расчета расстояний}

Алгоритм расчета расстояний представлен на \fullref{schemes/distance.png}

\putimage[0.8]{schemes/distance.png}{
  Блок-схема алгоритма расчета расстояний
}

\newpage
\section{Программное обеспечение}

\subsection{Аналитический обзор существующих программных технологий, применимых при решении поставленных задач}
\subsection{Архитектура разрабатываемого программного продукта}
\subsection{Язык программирования и инструментальные средства разработки}
\subsection{Технологии разработки ПО (моделирование разработки ПО, управление разработкой ПО, конфигурирование ПО, технологии тестирования ПО)}
\subsection{Описание структуры программного продукта}
\subsection{Описание интерфейса пользователя}

\newpage
\section{Оценка качества рещения}
\subsection{Тестирование ПО}
\subsection{Оценка качества программного продукта}
\subsection{Вычислительный эксперимент и анализ результатов}

\newpage
\section*{ЗАКЛЮЧЕНИЕ}
\addcontentsline{toc} {section} {ЗАКЛЮЧЕНИЕ}

\subsection*{Результаты работы}
\addcontentsline{toc} {subsection} {Результаты работы}

В представленной работе описан программный продукт для предотвращения пересечения стволов скважин.
Он позволяет осуществлять следующие действия:
импортировать данные измерительных устройств, визуализировать траектории стволов и оценивать расстояния между ними.

Такой программный продукт может найти применение в строительстве с использованием технологии искусственного замораживания
грунтов, а так же в бурении нефтяных и газовых скважин.

Для работы с данной программой требуется знания в области нефтяной промышленности, а также понимание принципов инклинометрии.
Требования к наличию знаний компьютерных навыков минимальна.

\subsection*{Выводы}
\addcontentsline{toc} {subsection} {Выводы}
??? - в голове что-то типа "в процессе выполнения ВКР я познал дзен и понял всю тщетность бытия"

\newpage
\section*{СПИСОК ИСПОЛЬЗОВАННЫХ ИСТОЧНИКОВ}
\addcontentsline{toc} {section} {СПИСОК ИСПОЛЬЗОВАННЫХ ИСТОЧНИКОВ}

\begin{thebibliography}{99}
  \bibitem{wiki} Web-портал http://ru.wikipedia.org/ - свободная энциклопедия.
  \bibitem{gost} ГОСТ 28195-89 – «Оценка качества программных средств».
  \bibitem{qt} Макс Шлее Qt 4.8. Профессиональное программирование на C++, — СПб.: БХВ-Петербург, 2012. — 894 стр.
  % \bibitem{metro} Портал http://www.metro.ru/library/stroitelstvo_metropolitenov/512/ - московское метро.
  % \bibitem{mountain} Портал http://www.mining-enc.ru/z/zamorazhivanie-gruntov/ - горная энциклопедия.
  % \bibitem{uhta} http://lib.ugtu.net/sites/default/files/books/2014/bliznyukov_v.yu_._raschyot_i_korrektirovanie_traektorii_skvazhiny_pri_burenii_2014.pdf
\end{thebibliography}

\newpage
\section*{ПРИЛОЖЕНИЯ}
\addcontentsline{toc}{section}{ПРИЛОЖЕНИЯ}

\subsection*{Программная документация}
\addcontentsline{toc}{subsection}{Программная документация}

\subsubsection*{Техническое задание}
\addcontentsline{toc}{subsection}{Техническое задание}

\subsubsection*{Руководство программиста}
\addcontentsline{toc}{subsection}{Руководство программиста}

\subsubsection*{Руководство пользователя}
\addcontentsline{toc}{subsection}{Руководство пользователя}

\end{document}
