\documentclass[12pt]{article}
\usepackage{fontspec}
\usepackage{array}
\usepackage{indentfirst}
\usepackage{fancyhdr}
\usepackage{ragged2e}
\usepackage{graphicx}
\usepackage{float}
\usepackage[labelsep=period]{caption}
\usepackage{tabularx}
\usepackage{longtable}
\usepackage{amsmath}



\graphicspath{{images/}}

\newcommand{\putimage}[3][1] {
  \begin{figure}[H]
  \center{\includegraphics[width=#1\linewidth]{#2}}
  \caption{#3}
  \label{ris:#2}
  \end{figure}
}

\newcommand{\fullref}[1] {
  \figurename \ref{ris:#1}
}

\newenvironment{ztable}[3]
{\begin{longtable}[H]{#1}
  \captionsetup{justification=raggedleft,singlelinecheck=false}
  \caption{#2 \label{tab:#3}} \\
}
{\end{longtable}}


\setmainfont[Mapping=tex-text]{Times New Roman}

\pagestyle{fancy}
\fancyhf{}
\fancyhead[R]{\thepage}
\renewcommand{\headrulewidth}{0pt}
\fancyheadoffset{0cm}

\renewcommand{\figurename}{Рис.}
\renewcommand{\tablename}{Таблица }
\renewcommand{\contentsname}{Оглавление}
\renewcommand\thesection{\arabic{section}.}
\renewcommand\thesubsection{\arabic{section}.\arabic{subsection}}
\renewcommand\theequation{\arabic{section}.\arabic{equation}}

\usepackage{geometry} % Меняем поля страницы
\geometry{left=3cm}% левое поле
\geometry{right=1cm}% правое поле
\geometry{top=2cm}% верхнее поле
\geometry{bottom=2cm}% нижнее поле

\sloppy
\begin{document}
\begin{titlepage}
\newpage

\begin{center}
Федеральное государственное бюджетное образовательное учреждение\\
высшего профессионального образования \\
\vspace{0.5cm}
УФИМСКИЙ ГОСУДАРСТВЕННЫЙ АВИАЦИОННЫЙ ТЕХНИЧЕСКИЙ УНИВЕРСИТЕТ\\
\vspace{0.5cm}
ФАКУЛЬТЕТ  ИНФОРМАТИКИ  И  РОБОТОТЕХНИКИ \\
\vspace{0.5cm}
КАФЕДРА  ВЫЧИСЛИТЕЛЬНОЙ  МАТЕМАТИКИ  И  КИБЕРНЕТИКИ\\
\vspace{0.5cm}
Направление 231000 – Программная инженерия
\end{center}

\vspace{2em}

\begin{center}
\Large {ВЫПУСКНАЯ КВАЛИФИКАЦИОННАЯ РАБОТА}

\vspace{2.5em}

Тема: Программа для расчета и предупреждения пересечения стволов нефтяных скважин
\end{center}

\vspace{3em}
\noindent
\begin{tabularx}{\textwidth}{|l|c|X|X|}
\hline
& ФИО & Подпись & Дата\\
\hline
Студент & Синявский Г. Н. &&\\
\hline
Руководитель работы & Еникеева К. Р. &&\\
\hline
Консультант & Еникеева К. Р. &&\\
\hline
Контроль программного продукта &&&\\
\hline
Председатель комиссии по предзащите &&&\\
\hline
Рецензент &&& \\
\hline
\end{tabularx}

\vspace{2em}
\center Допущен к защите\\
Зав. кафедрой  ВМК, д.т.н., проф.\\
\underline{\hspace{5cm}} Н.И. Юсупова\\
“\underline{\hspace{1cm}}”\underline{\hspace{5cm}} 2015 г.

\vspace{\fill}

\begin{center}
УФА~--- 2015г.
\end{center}

\end{titlepage}

\newpage

\begin{center}
Федеральное государственное бюджетное образовательное учреждение\\
высшего профессионального образования \\
\vspace{0.5cm}
УФИМСКИЙ ГОСУДАРСТВЕННЫЙ АВИАЦИОННЫЙ ТЕХНИЧЕСКИЙ УНИВЕРСИТЕТ\\
\vspace{0.5cm}
ФАКУЛЬТЕТ  ИНФОРМАТИКИ  И  РОБОТОТЕХНИКИ \\
\vspace{0.5cm}
КАФЕДРА  ВЫЧИСЛИТЕЛЬНОЙ  МАТЕМАТИКИ  И  КИБЕРНЕТИКИ\\
\vspace{1cm}
\end{center}
\vspace{0.5cm}
\begin{flushright}
"УТВЕРЖДАЮ" \\
Зав. кафедрой  ВМК, д.т.н., проф.\\
\vspace{0.5cm}
\underline{\hspace{5cm}} Н.И. Юсупова\\
«08»    октября  2015  г.
\end{flushright}
\vspace{1cm}

\begin{center}
  \Large{ ЗАДАНИЕ }
\vspace{0.5cm}

на подготовку выпускной квалификационной работы
\end{center}

студента Синявского Глеба Николаевича

\begin{enumerate}
  \item Тема работы - Программа для предупреждения пересечения стволов скважин\\
  ( утверждена  распоряжением  по  факультету  №  72/1    от “  15 октября   2015  г )
  \item Срок  представления  работы  “20”     января     2016  г.
  \item Описание задачи\\
    Необходимо разработать программный продукт, позволяющий усреднять и визуализировать замеры стволов скважин, а так же
    позволять оценивать расстояния между ними.
  \item Математическая часть\\
  В качестве математических моделей, принятых для реализации в рамках программного продукта,
  необходимо использовать уравнения для расчета координат ствола методом среднего угла.
  \item Спецификация входных и выходных данных\\
  Входные данные - csv-файлы, содержащие результаты замера ствола скважины. Выходные - визуализация скважины в пространстве, визуализации оценки расстояний
  между стволами.
  \item Применяемые инструментальные средства\\
  Библиотека построение графического интерфейса - Qt. СУБД - SQLite. Библиотека\\визуализации - MathGL.
  \item Особые условия эксплуатации программного продукта\\
  Основная ОС для запуска программного продукта - Windows 7 и старше, но продукт должен разрабатываться как кроссплатформенный и иметь возможность запуска
  под управлением ОС Linux.
  \item Дополнительные условия\\
  Продукт должен иметь возможность импортировать csv произвольного формата, для этого должен быть разработан мастер импорта, позволяющий
  выбирать диапазон ячеек таблицы и указывать их тип.
\end{enumerate}

\vspace{\fill}

Руководитель работы \underline{\hspace{5cm}}

Консультант \underline{\hspace{6.5cm}}
\begin{center}
  Дата  выдачи   «08»  октября       2015 г.
\end{center}
\newpage

\tableofcontents % это оглавление, которое генерируется автоматически
\newpage
\section*{ Аннотация}
\addcontentsline{toc} {section} {Аннотация}
%TODO: написать аннотацию

\newpage
\section*{Введение}
\addcontentsline{toc}{section}{Введение}
%TODO: написать введение
\subsection*{Описание предметной области}
\addcontentsline{toc}{subsection}{Описание предметной области}

\subsection*{Мотивация, актуальность проблемы}
\addcontentsline{toc}{subsection}{Мотивация, актуальность проблемы}

\subsection*{Цели, задачи ВКР}
\addcontentsline{toc}{subsection}{Цели, задачи ВКР}
Целью дипломной работы является разработка программиного обеспечения, позволяющего визуализировать, усреднять и производить анализ замеров стволов нефтяных скважин,
на основании данных, полученных с измерительного оборудования. Для достижения поставленной цели необходимо решить следующие задачи:
\begin{itemize}
  \item провести анализ существующих программных продуктов;
  \item разработка функциональной и информационной моделей, программного обеспечения;
  \item разработка модуля импорта данных
  \item разработка системы управления содержимым БД и усреднения замеров
  \item разработка модуля визуализации замеров
  \item разаботка модуля рассчетов расстояний между стволами
  \item разработка модуля визуализации расстояний между стволами
\end{itemize}

\subsection*{Содержание работы по главам}
\addcontentsline{toc}{subsection}{Содержание работы по главам}

\newpage
\section{Анализ проблемы и постановка задачи}
\subsection{Анализ предметной области}
Бурение скважин — это процесс сооружения направленной цилиндрической горной выработки в земле, диаметр "D" которой мал по сравнению с её длиной по стволу "H",
без доступа человека на забой. Начало скважины на поверхности земли называют устьем, дно — забоем, а стенки скважины образуют ее ствол.

Кустовое бурение — сооружение скважин (в основном наклонно направленных), устья которых группируются на близком расстоянии друг от друга на общей ограниченной
площадке (основании). Применяется при разработке месторождений под застроенными участками, при разработке нефтяных и газовых месторождений в определённых
климатических условиях (например, в зимний период, когда наблюдается большой снеговой покров, или весной во время распутицы и значительных паводков),
на территории с сильно пересечённым рельефом местности или в пределах акваторий.

В настоящее время практически все эксплуатационные скважины бурятся кустовым методом, когда устья нескольких скважин в кусте расположены близко друг к другу (4–5
м) на одной технологической площадке, а забои находятся в узлах сетки разработки. Число скважин в кусте колеблется от  2 до нескольких десятков.

Искусственное искривление скважин применяется с целью:
\begin{itemize}
  \item добычи нефти и газа из труднодоступных участков, занятых на поверхности промышленными и жилыми объектами, оврагами, горами, реками, озерами, болотами, лесами, морями;
  \item экономии отводимых под строительство скважин плодородных земельных участков, лесов и др.;
  \item экономии затрат на строительство оснований, подъездных путей, линий электропередач, связи, трубопроводов;
  \item сокращения средств и времени на строительно-монтажные работы и обслуживание при эксплуатации скважин с близко расположенными устьями;
  \item обхода зон катастрофических поглощений, обвалов и аварий в стволе скважины;
  \item вскрытия продуктивных пластов, залегающих под пологим сбросом или между двумя параллельными сбросами;
  \item проходки стволов на нефтяные пласты, залегающие под соляными куполами, в связи с трудностью бурения через них (соль «плывет», срезает бурильные и обсадные колонны);
  \item бурения стволов для глушения открытых фонтанов и тушения пожаров;
  \item перебуривания части ствола скважины;
  \item вскрытия продуктивного пласта под определенным углом для увеличения поверхности дренажа и дебита скважины;
  \item многозабойного вскрытия продуктивного пласта.
\end{itemize}

Помимо искусственного искривления скважин так же имеет место быть самопроизвольное искривление, связанное с геологическими, техническими и технологическими факторами.

\textbf{Геологические факторы:}
\begin{itemize}
  \item перемежаемость по твердости - чередование мягких и твердых горных пород;
  \item жеоды - инородные тела в составе горное породы, отличающиеся по твердости;
\end{itemize}

\textbf{Технические факторы:}
\begin{itemize}
  \item несоосность вышки  относительно осей стола ротора  и шахтового направления;
  \item негоризонтальность стола ротора;
  \item использование искривленных труб (ведущих и бурильных) и труб, у которых резьбы нарезаны под углом;
  \item эксцентричное забуривание нижележащего участка скважины.
\end{itemize}

\textbf{Технологические факторы:}
\begin{itemize}
  \item влияние осевой нагрузки;
  \item влияние частоты вращения бурильной колонны.
\end{itemize}

\textbf{Методы предупреждения самопроизвольного искривления скважин}
\begin{itemize}
  \item для предупреждения естественного искривления скважин необходимо исключить или уменьшить действие управляемых технических факторов и нейтрализовать действие неуправляемых геологических условий;
  \item технические причины искривлений должны быть устранены до начала бурения скважины;
  \item действие технологических причин искривления могут быть сведены к минимуму центрированием низа бурильной колонны, увеличением его жесткости, регулированием осевой нагрузки;
  \item цель центрирования нижней части бурильной колонны - препятствовать отклонению оси долота от оси скважины;
  \item увеличение жесткости и массы нижней части бурильной колонны повышает устойчивость к изгибу, уменьшает длину сжатой части, позволяет использовать повышенные нагрузки на долото;
  \item для компенсации геологических причин искривления (наклонно-залегающие анизотропные породы) можно использовать методы наклонно направленного разбуривания ствола в направлении, противоположном естественному искривлению;
  \item метод буровых трасс – перенос устья скважины по азимуту и величине смещения при самопроизвольном искривлении скважин.
\end{itemize}

В связи с неуправляемым действием геологических условий, при бурении следует контролировать данные инклинометрии и вовремя принимать решения об искусственном искривлении
ствола для предотвращения пересечения скважин и минимизации отхода траектории от проектной.

Поскольку глубина промерзания грунта вокруг ствола замораживающей скважины равна примерно 70см, то, с учетом допуска и погрешности, такие скважины должны буриться
на расстоянии примерно 119см друг от друга по всей длине ствола, что требует дополнительного контроля за данным инклинометрии и увеличивает вероятность их пересечения.

В настоящей работе описан программный продукт, позволяющий анализировать данные инклинометрии и оценивать расстояния между стволами по всей их длине, тем
самым упрощая принятие решения об искусственном искривлении стволов и предупреждении их пересечения.


\subsection{Содержательная постановка проблемы}
Целью выпускной квалификационной работы является разработка программного обеспечения для предупреждения пересечения стволов скважин. Входной информацией для
программы являются данные измерительного оборудования, выходной - графическое представление стволов и графическое представление расстояний между ними.


\subsection{Формальная постановка задачи}
Формальной постановке задачи соответствует контекстная диаграмма методологии IDEF0, описывающая входные и выходные данные, управляющие воздействия и механизмы,
влияющие на систему в целом, приведённая на \fullref{fm/a1.png}
\putimage{fm/a1.png}{
  Общая контекстная диаграмма
}

\subsection{Структура решения задачи, декомпозиция задачи на подзадачи}
Общую задачу можно декомпозировать и выделить следующие подзадачи:
\begin{itemize}
  \item импорт данных измерительного оборудования;
  \item усреднение траекторий;
  \item визуализация стволов;
  \item визуализация расстояний.
\end{itemize}

\putimage{fm/a2.png}{
  Структура решения общей задачи
}

\newpage
\section{Математическое и информационное обеспечение }

??? - тут как-то нужно затолкать алгоритм расчета траектории, то есть перевод из данных инклинометрии в декартову систему координат,
тот самый метод среднего угла. Еще можно описать алгоритм усреднения замеров и расчета расстояний между стволами. Алгоритмы я могу описать,
но как и куда их тут затолкать не представляю

\subsection{Классификация подзадач (отнесение подзадач к классу задач)}
???

\subsection{Математические модели подзадач (где применимо)}
???

\subsection{Методы решения подзадач (где применимо)}
???

\subsection{Информационные модели для подзадач (где применимо)}
???

\subsection{Алгоритмы и структуры данных для подзадач}

\newpage
\section{Программное обеспечение}

\subsection{Аналитический обзор существующих программных технологий, применимых при решении поставленных задач}

Существует два основных подхода к разработке программного обеспечения: процедурный и объектно-ориентированный.
Каждый из них имеет свои преимущества и недостатки.

При использовании процедурного подхода, обычно появляется огромное количество глобальных переменных, и структурирование
происходит на уровне модулей (файлов).

Объектно-ориентированное или объектное программирование — парадигма программирования, в которой основными концепциями
являются понятия объектов и классов.

Класс — это тип, описывающий устройство объектов. Понятие «класс» подразумевает некоторое поведение и способ представления.
Понятие «объект» подразумевает нечто, что обладает определённым поведением и способом представления. Говорят, что объект — это экземпляр класса. Класс можно сравнить с чертежом, согласно которому создаются объекты. Обычно классы разрабатывают таким образом, чтобы их объекты соответствовали объектам предметной области.

Класс является описываемой на языке терминологии (пространства имён) исходного кода моделью ещё не существующей сущности,
т. е. объекта.

Объект — сущность в адресном пространстве вычислительной системы, появляющаяся при создании экземпляра класса (например,
после запуска результатов компиляции (и связывания) исходного кода на выполнение).

Перечислим некоторые достоинства ООП:
\begin{itemize}
  \item классы позволяют проводить конструирование из полезных компонент, обладающих простыми инструментами,
    что дает возможность абстрагироваться от деталей реализации.
  \item данные и операции вместе образуют определенную сущность и они не «размазываются» по всей программе,
    как это нередко бывает в случае процедурного программирования.
  \item локализация кода и данных улучшает наглядность и удобство сопровождения программного обеспечения.
  \item инкапсуляция информации защищает наиболее критичные данные от несанкционированного доступа.
\end{itemize}

Model-view-controller (MVC, «модель-представление-контроллер», «модель-вид-контроллер») — схема использования нескольких
шаблонов проектирования, с помощью которых модель приложения, пользовательский интерфейс и взаимодействие с пользователем
разделены на три отдельных компонента таким образом, чтобы модификация одного из компонентов оказывала минимальное
воздействие на остальные. Данная схема проектирования часто используется для построения архитектурного каркаса,
когда переходят от теории к реализации в конкретной предметной области.

Предметно-ориентированное проектирование (реже проблемно-ориентированное, англ. Domain-driven design, DDD) —
это набор принципов и схем, помогающих разработчикам создавать изящные системы объектов. При правильном применении
оно приводит к созданию программных абстракций, которые называются моделями предметных областей. В эти модели входит
сложная бизнес-логика, устраняющая промежуток между реальными условиями области применения продукта и кодом.

\subsection{Архитектура разрабатываемого программного продукта}

Для создания программного продукта был применен объектно-ориентированный подход к программированию,
т.к. решение данной задачи требует жесткого контроля над пользователем при работе с системы.
Объектно-ориентированный подход применялся там где, происходит сбор информации т.к. необходимо четко определить формат данных.

Qt использует концепцию MVC для отделения данных(Model) от их представления(View) в таких компонентах, как QTreeView, QListView и
QTableView \cite{qt}.

\subsection{Язык программирования и инструментальные средства разработки}

\subsubsection{Язык C++}
На данный момент, C++ остаётся одним из самых популярных и производительных языков программирования и применяется практически во всех прикладных областях
программирования, от низкоуровневого программирования для микроконтроллеров, до высокопроизводительных серверных приложений и компьютерных игр.

\subsubsection{SQLite}
SQLite — это встраиваемая кроссплатформенная СУБД, которая поддерживает достаточно полный набор команд SQL и доступна в исходных кодах (на языке C). На данный
момент является самой популярной встраиваемой СУБД. Применяется как на персональный компьютерах, так и в мобильных ОС и "умных" телевизорах.

\subsubsection{Qt}
Qt — кроссплатформенный инструментарий разработки ПО на языке программирования C++, доступен в исходных текстах. Позволяет создавать кроссплатформенные приложения с богатыми возможностями
графического интерфейса, работой с сетью, мультимедиа, БД и 3D-графикой. В окружении каждой поддерживаемой ОС будет выглядеть максимально похоже на "родные" приложения
системы.

\subsubsection{MathGL}
MathGL — кроссплатформенная библиотека для визуализации данных. Имеет интеграцию с Qt.

\subsubsection{Обоснованность выбора технологий}
На данный момент указанные технологии являются единственным способом, как выполнить требования о кроссплатформенности, так и получить лёгкий в поддержке
продукт, базирующийся на надёжных и поддерживаемых библиотеках.

\subsection{Описание структуры программного продукта}
Структура программного продукта изображения на \fullref{schemes/architecture.png}

\putimage[1]{schemes/architecture.png} {
  Структура программного продукта
}

\subsection{Описание интерфейса пользователя}
После запуска программы открывается основное окно, главными элементами которого являются дерево, представляющее иерархию предметной отрасли,
и таблица редактирования выбранного в дереве уровня, \fullref{manual/10.jpg}

\putimage[1]{manual/10.jpg} {
  Главное окно программы
}

Интерфейс пользователя построен с использованием стандартных элементов и выглядит аналогично на всех поддерживаемых платформах. Вверху окна находится
главное меню, в котором скрыты настройки, переключения языка и другие редко используемые функции. Под главным меню располагается панель инструментов,
на которой расположены более популярные функции "График", "Импорт", "Предупреждение пересечения стволов" и "Создать усредненный".

На \fullref{manual/19.jpg} изображен диалог с профилем ствола, изображение можно вращать мышью и настраивать отображение с помощью настроек справа.

\putimage[1]{manual/19.jpg} {
  Диалог профиля ствола
}

Подробно интерфейс пользователя описан в приложении "Руководство пользователя".

\newpage
\section{Оценка качества решения}

\subsection{Тестирование ПО}

Тестирование является важной и обязательной частью процесса разработки.

Процесс тестирования можно разделить на 3 этапа:
\begin{itemize}
  \item проверка в нормальных условиях;
  \item проверка в экстремальных условиях;
  \item проверка в исключительных ситуациях.
\end{itemize}

\textbf{Тестирование в нормальных условиях}

При проверке в нормальных условиях программа функционировала соответствующим образом:
введенные данные были без потерь сохранены в базе данных в нужном формате и в результате запросов были выданы верные сведения.
\putimage{manual/3.jpg}{Ввод корректных параметров подрядчика}
\putimage{manual/4.jpg}{Созданный подрядчик}

\textbf{Тестирование в экстремальных условиях}

Проводилась проверка на ввод нулевых и отсутствующих параметров. Программа не позволяет ввести неверные значения, т.н.
"защита от дурака" (Рис. 26).

\putimage{tests/1.png}{Недоступная кнопка ОК при попытка создать заказчика без названия}

\textbf{Тестирование в исключительных ситуациях}

Тестирование устойчивости программы при вводе неверных данных проводилось с самого начала разработки.
Построение интерфейса программы предусматривает предотвращение возможности совершения пользователем действий,
приводящих к исключительным ситуациям.

Практически невозможна ситуация, когда в результате сбоя разработанное ПО выйдет из-под контроля и
нарушит целостность исходных данных, системы или других прикладных программ.

\textbf{Анализ тестирования}

Тестирование, проведенное в различных условиях, подтверждает работоспособность программы. Возможно, в процессе эксплуатации программы потребуются некоторые ее доработки.


\subsection{Оценка качества программного продукта}

\textbf{Метрическая оценка качества программного продукта.}

В данной части дипломной работы проводится оценка качества программного продукта согласно ГОСТ 28195-89.

\textbf{Определение подкласса программных средств}

Данное программное средство относится к подклассу 509 – Прочие ПС.

\textbf{Показатели надёжности программного средства}

\begin{ztable}{|p{2cm}|p{10cm}|p{2cm}|p{2cm}|}{ Оценочные элементы фактора “Надёжность ПС”}{reliability_table}
    \hline
    Код элемента & Наименование & Метод оценки & Оценка\\

    \endhead

    \hline
    \multicolumn{4}{|c|}{ \textbf{Средства восстановления при ошибках на входе}} \\

    \hline
    Н0101 & Наличие требований к программе по устойчивости функционирования при наличии ошибок во входных данных  & Экспертный & 1 \\

    \hline
    Н0102 & Возможность обработки ошибочных ситуаций & То же & 1 \\

    \hline
    Н0103 & Полнота обработки ошибочных ситуаций & » & 1 \\

    \hline
    Н0104 & Наличие тестов для проверки допустимых значений входных данных & » с& 0 \\

    \hline
    Н0105 & Наличие системы контроля полноты входных данных & » & 0 \\

    \hline
    Н0106 & Наличие средств контроля корректности входных данных & » & 1 \\

    \hline
    Н0107 & Наличие средств контроля непротиворечивости входных данных & » & 0 \\

    \hline
    Н0108 & Наличие проверки параметров и адресов по диапазону их значений & » & 1 \\
    Н0109 & Наличие обработки граничных результатов & » & 1 \\

    \hline
    Н0110 & Наличие обработки неопределенностей & » & 0,6 \\

    \hline
    &&& 0,8 \\

    \hline
    \multicolumn{4}{|c|}{ \textbf{Средства восстановления при сбоях оборудования}} \\

    \hline
    Н0201 & Наличие требований к программе по восстановлению процесса выполнения в случае сбоя операционной системы, процессора, внешних устройств & » & 0 \\

    \hline
    Н0202 & Наличие требований к программе по восстановлению результатов при отказах процессора, ОС & » & 1 \\

    \hline
    Н0203 & Наличие средств восстановления процесса в случае сбоев оборудования & » & 0 \\

    \hline
    Н0204 & Наличие возможности разделения по времени выполнения отдельных функций программ & » & 1 \\

    \hline
    Н0205 & Наличие возможности повторного старта с точки останова & » & 1 \\

    \hline
    &&& 0,6 \\

    \hline
    \multicolumn{4}{|c|}{ \textbf{Реализация управления средствами восстановления}} \\

    \hline
    Н0301  & Наличие централизованного управления процессами, конкурирующими из-за ресурсов & » & 1 \\

    \hline
    Н0302  & Наличие возможности автоматически обходить ошибочные ситуации в процессе вычисления & » & 0 \\

    \hline
    Н0303 & Наличие средств, обеспечивающих завершение процесса решения в случае помех & » & 1 \\

    \hline
    Н0304  & Наличие средств, обеспечивающих выполнение программы в сокращенном объеме в случае ошибок или помех & » & 0 \\

    \hline
    Н0305 & Показатель устойчивости к искажаемым воздействиям& Расчетный & 0 \\

    \hline
    &&& 0,4 \\

    \hline
    \multicolumn{4}{|c|}{ \textbf{Функционирование в заданных режимах}} \\

    \hline
    Н0401 & Вероятность безотказной работы & То же & 1 \\

    \hline
    \multicolumn{4}{|c|}{ \textbf{Обеспечение обработки заданного объема информации}} \\

    \hline
    Н0501 & Оценка по среднему времени восстановления & » & 1 \\

    \hline
    Н0502 & Оценка по продолжительности преобразования входного набора данных в выходной & » & 1 \\

    \hline
    &&& 1 \\
    \hline
\end{ztable}

\textbf{Показатели сопровож­дения}

\begin{ztable}{|p{2cm}|p{10cm}|p{2cm}|p{2cm}|}{Оценочные элементы фактора “Сопровождаемость ПС”}{support_table}
    \hline
    Код элемента & Наименование & Метод оценки & Оценка\\

    \endhead

    \hline
    \multicolumn{4}{|c|}{\textbf{Простота архитектуры проекта}} \\

    \hline
    С0101 & Наличие модульной схе­мы программы & Экспертный & 1 \\

    \hline
    С0102 & Оценка программы по числу уникальных модулей & » & 1 \\

    &&& 1 \\

    \hline
    \multicolumn{4}{|c|}{\textbf{Сложность архитектуры проекта}} \\

    \hline
    С0201 & Наличие ограничений на размеры модуля & » & 0 \\

    \hline
    \multicolumn{4}{|c|}{\textbf{Межмодульные связи}} \\


    \hline
    С030 & Наличие требований к не­зависимости модулей про­граммы от типов и форма­тов выходных данных & » & 0 \\

    \hline
    С0301& Наличие проверки кор­ректности передаваемых данных & » & 1 \\

    \hline
    С0302 & Оценка простоты прог­раммы по числу точек вхо­да и выхода  & Расчетный & 0,01 \\

    \hline
    С0303 & Осуществляется ли пере­дача результатов работы модуля через вызывающий его модуль  & Экспертный & 1 \\

    \hline
    С0304& Осуществляется ли конт­роль за правильностью дан­ных, поступающих в вызы­вающий модуль от вызыва­емого & » & 1 \\



    \hline
    &&& 0,6 \\

    \hline
    \multicolumn{4}{|c|}{\textbf{Экспертиза принятой системы идентификации}} \\

    \hline
    С0601 & Использование при пост­роении программ метода структурного программиро­вания & » & 1 \\

    \hline
    С0602& Соблюдение принципа разработки программы сверху вниз & » & 1 \\

    \hline
    С0603 & Оценка программы по числу циклов с одним вхо­дом и одним выходом & » & 1 \\

    \hline
    С0604 & Оценка программы по числу циклов & » & 1 \\

    \hline
    &&& 1 \\

    \hline
    \multicolumn{4}{|c|}{\textbf{Комментарии логики программ проекта}} \\


    \hline
    С0801 & Наличие комментариев ко всем машинозависимым час­тям программы & » & 0 \\

    \hline
    С0802 & Наличие комментариев к машинозависимым операто­рам программы & » & 0 \\

    \hline
    С0803 & Наличие комментариев в точках входа и выхода про­граммы & » & 1 \\

    \hline
    &&& 0,3 \\

    \hline
    \multicolumn{4}{|c|}{\textbf{Оформление текста программ}} \\

    \hline
    С0901 & Соответствие комментари­ев принятым соглашениям & » & 0 \\

    \hline
    С0902 & Наличие комментариев-за­головков программы с ука­занием ее структурных и функциональных характе­ристик & » & 0 \\

    \hline
    С0903& Оценка ясности и точнос­ти описания последователь­ности функционирования всех элементов программы& » & 0 \\

    \hline
    &&& 0 \\

    \hline
    \multicolumn{4}{|c|}{\textbf{Простота кодирования}} \\


    \hline
    С1001 & Используется ли язык высокого уровня & » & 1 \\

    \hline
    С1002 & Оценка простоты прог­раммы по числу переходов по условию & Расчетный & 0,3 \\

    \hline
    &&& 0,6 \\

    \hline
\end{ztable}

\textbf{Показатели удобства применения}

\begin{ztable}{|p{2cm}|p{10cm}|p{2cm}|p{2cm}|}{Оценочные элементы фактора “Удобство применения ПС”}{convenience_table}
    \hline
    Код элемента & Наименование & Метод оценки & Оценка\\

    \endhead

    \hline
    \multicolumn{4}{|c|}{\textbf{Освоение работы ПС}} \\


    \hline
    У0101 & Возможность освоения програм­мных средств по документации & Экспертный & 1 \\

    \hline
    У0102 & Возможность освоения ПС на конт­рольном  примере при  помощи  ЭВМ & То же & 1 \\

    \hline
    У0103 & Возможность поэтапного освоения ПС & » & 1 \\

    \hline
    &&& 1 \\

    \hline
    \multicolumn{4}{|c|}{\textbf{Документация для освоения}} \\

    \hline
    У0201 & Полнота и понятность документа­ции для освоения & » & 1 \\

    \hline
    У0202 & Точность документации для освое­ния & » & 1 \\

    \hline
    У0203 & Техническое исполнение докумен­тации & » & 0,4 \\

    \hline
    &&& 0,8 \\


    \hline
    \multicolumn{4}{|c|}{\textbf{Полнота пользовательской документации}} \\


    \hline
    У0301 & Наличие краткой аннотации & » & 1 \\

    \hline
    У0302 & Наличие описания решаемых задач & » & 1 \\

    \hline
    У0303 & Наличие описания структуры  функ­ции ПС & » & 1 \\

    \hline
    У0304  & Наличие описания основных функ­ций ПС & » & 1 \\

    \hline
    У0306  & Наличие описания частных функ­ций & » & 1 \\

    \hline
    У0307  & Наличие описания алгоритмов & » & 0 \\

    \hline
    У0308  & Наличие описания межмодульных интерфейсов & » & 0 \\

    \hline
    У0309  & Наличие описания пользовательских интерфейсов & » & 1 \\

    \hline
    У0310  & Наличие описания входных и вы­ходных данных & » & 1 \\

    \hline
    У0311 & Наличие описания диагностических сообщений & » & 0 \\

    \hline
    У0312  & Наличие описания основных харак­теристик ПС & » & 1 \\

    \hline
    У0314  & Наличие описания программной среды функционирования ПС & » & 1 \\

    \hline
    У0315  & Достаточность документации для ввода ПС в эксплуатацию & » & 1 \\

    \hline
    У0316 & Наличие информации технологии переноса  для  мобильных программ & » & 0 \\

    \hline
    &&& 0,7 \\

    \hline
    \multicolumn{4}{|c|}{\textbf{Точность пользовательской документации}} \\

    \hline
    У0401& Соответствие оглавления содержа­нию документации & » & 1 \\

    \hline
    У0402& Оценка оформления документации & » & 1 \\

    \hline
    У0403& Грамматическая правильность из­ложения документации & » & 1 \\

    \hline
    У0404& Отсутствие противоречий & » & 1 \\

    \hline
    У0405& Отсутствие неправильных ссылок & » & 1 \\

    \hline
    У0406& Ясность формулировок и описаний & » & 1 \\

    \hline
    У0407& Отсутствие неоднозначных форму­лировок и описаний & » & 1 \\

    \hline
    У0408& Правильность использования тер­минов & » & 1 \\

    \hline
    У0409 & Краткость,  отсутствие лишней  де­тализации & » & 1 \\

    \hline
    У0410 & Единство формулировок & » & 1 \\

    \hline
    У0411 & Единство обозначений & » & 1 \\

    \hline
    У0412 & Отсутствие ненужных повторений & » & 1 \\

    \hline
    У0413 & Наличие нужных объяснений & » & 1 \\

    \hline
    &&& 1 \\

    \hline
    \multicolumn{4}{|c|}{\textbf{Понятность пользовательской документации}} \\

    \hline
    У0501  & Оценка стиля изложения  & » & 1 \\

    \hline
    У0502  & Дидактическая разделенность & » & 1 \\

    \hline
    У0503  & Формальная разделенность & » & 1 \\

    \hline
    У0504  & Ясность логической структуры  & » & 1 \\

    \hline
    У0505  & Соблюдение стандартов и правил изложения в документации  & » & 1 \\

    \hline
    У0506 & Оценка по числу ссылок вперед в тексте документов  & » & 0 \\

    \hline
    &&& 0,8 \\

    \hline
    \multicolumn{4}{|c|}{\textbf{Техническое исполнение пользовательской документации}} \\

    \hline
    У0601 & Наличие оглавления & » & 1 \\

    \hline
    У0602 & Наличие предметного указателя & » & 0 \\

    \hline
    У0603 & Наличие перекрестных ссылок & » & 0 \\

    \hline
    У0604 & Наличие всех требуемых разделов & » & 1 \\

    \hline
    У0605& Соблюдение непрерывности нуме­рации страниц документов & » & 1 \\

    \hline
    У0606 & Отсутствие незаконченных разделов абзацев, предложений & » & 1 \\

    \hline
    У0607 & Наличие всех рисунков, чертежей, формул, таблиц & » & 1 \\

    \hline
    У0608 & Наличие всех строк и примечаний & » & 1 \\

    \hline
    У0609 & Логический порядок частей внутри главы & » & 1 \\

    \hline
    &&& 0,8 \\

    \hline
    \multicolumn{4}{|c|}{\textbf{Прослеживание вариантов пользовательской документации}} \\

    \hline
    У0701 & Наличие полного перечня докумен­тации & » & 1 \\

    \multicolumn{4}{|c|}{\textbf{Эксплуатация}} \\

    \hline
    У0801 & Уровень языка общения пользова­теля с программой & » & 1 \\

    \hline
    У0802  & Легкость и быстрота загрузки и запуска программы & » & 1 \\

    \hline
    У0803 & Легкость и быстрота завершения работы программы & » & 1 \\

    \hline
    У0804  & Возможность распечатки содержи­мого программы & » & 0,7 \\

    \hline
    У0805  & Возможность приостанови и пов­торного запуска работы без потерь информации & » & 1 \\

    \hline
    &&& 0,9 \\

    \hline
    \multicolumn{4}{|c|}{\textbf{Управление меню}} \\


    \hline
    У0901 & Соответствие меню требованиям пользователя & » & 1 \\

    \hline
    У0902 & Возможность прямого перехода вверх и вниз  по многоуровневому ме­ню (пропуск уровней) & » & 1 \\

    \hline
    &&& 1 \\



    \hline
    \multicolumn{4}{|c|}{\textbf{Функция Help}} \\

    \hline
    У1001 & Возможность управления подроб­ностью  получаемых выходных дан­ных  & » & 1 \\

    \hline
    У1002 & Достаточность полученной инфор­мации для продолжения работы  & » & 1 \\

    \hline
    &&& 1 \\

    \hline
    \multicolumn{4}{|c|}{\textbf{Управление данными}} \\

    \hline
    У1101 & Обеспечение удобства ввода дан­ных & » & 1 \\

    \hline
    У1102 & Легкость восприятия & » & 1 \\

    \hline
    &&& 1 \\

    \hline
    \multicolumn{4}{|c|}{\textbf{Рабочие процедуры}} \\

    \hline
    У1201 & Обеспечение программой выполне­ния предусмотренных рабочих про­цедур & » & 1 \\

    \hline
    У1202 & Достаточность информации, выда­ваемой программой для составления дополнительных процедур & » & 1 \\

    \hline
    &&& 1 \\

    \hline
\end{ztable}

\textbf{Показатели эффективности}

\begin{ztable}{|p{2cm}|p{10cm}|p{2cm}|p{2cm}|}{Оценочные элементы фактора “Эффективность ПС”}{efficiency_table}
    \hline
    Код элемента & Наименование & Метод оценки & Оценка\\

    \endhead

    \hline
    \multicolumn{4}{|c|}{ \textbf{Уровень автоматизации}} \\

    \hline
    Э0101 & Проблемно-ориентированные функции & Экспертный & 1 \\

    \hline
    Э0102 & Машинно-ориентированные функции & То же & 1 \\

    \hline
    Э0103 & Функции ведения и управления & » & 1 \\

    \hline
    Э0104 & Функции ввода/вывода & » & 1 \\

    \hline
    Э0105 & Функции защиты и проверки данных & » & 0 \\

    \hline
    Э0106 & Функции защиты от несанкционированного доступа & » & 1 \\

    \hline
    Э0107 & Функции контроля доступа & » & 1 \\

    \hline
    Э0108 & Функции защиты от внесения изменений & » & 1 \\

    \hline
    Э0109 & Наличие соответствующих границ функциональных областей & » & 1 \\

    \hline
    Э0110 & Число знаков после запятой в результатах вычислений & » & 1 \\

    \hline
    &&& 0,9 \\

    \hline
    \multicolumn{4}{|c|}{ \textbf{Временная эффективность}} \\

    \hline
    Э0201 & Время выполнения программ & » & 1 \\

    \hline
    Э0202 & Время реакции и ответов & » & 1 \\

    \hline
    Э0203 & Время подготовки & » & 1 \\

    \hline
    Э0205 & Затраты времени на защиту данных  & » & 0 \\

    \hline
    Э0206 & Время компиляции  & » & 1 \\

    \hline
    &&& 0,8 \\

    \hline
    \multicolumn{4}{|c|}{ \textbf{Ресурсоемкость}} \\


    \hline
    Э0301 & Требуемый объем  внутренней  памяти  & » & 1 \\

    \hline
    Э0302 & Требуемый объем  внешней  памяти  & » & 1 \\

    \hline
    Э0303 & Требуемые периферийные устройства  & » & 1 \\

    \hline
    Э0304 & Требуемое базовое программное обеспечение  & » & 1 \\

    \hline
    &&& 1 \\


    \hline
\end{ztable}

\textbf{Показатели универсальности}

\begin{ztable}{|p{2cm}|p{10cm}|p{2cm}|p{2cm}|}{Оценочные элементы фактора “Универсальность ПС”}{versality_table}
    \hline
    Код элемента & Наименование & Метод оценки & Оценка\\

    \endhead

    \hline
    \multicolumn{4}{|c|}{\textbf{Зависимость от используемого комплекса технических средств}} \\

    \hline
    Г0701 & Оценка зависимости программ от ёмкости оперативной памяти ЭВМ  & » & 1 \\

    \hline
    Г0702 & Оценка зависимости временных характеристик программы от скорости вычислений ЭВМ  & » & 1 \\

    \hline
    Г0703 & Оценка зависимости функционирования программы от числа внешних запоминающих устройств и их общей емкости  & » & 0 \\

    \hline
    Г0704 & Оценка зависимости функционирования программы от специальных устройств ввода-вывода & » & 1 \\

    \hline
    &&& 0,7 \\

    \hline
    \multicolumn{4}{|c|}{\textbf{Зависимость от базового программного обеспечения}} \\

    \hline
    Г0801 & Применение специальных языков программирования & » & 1 \\

    \hline
    Г0802 & Оценка зависимости программы от программ  операционной системы  & » & 1 \\

    \hline
    Г0803 & Зависимость от других программных средств & » & 1 \\

    \hline
    &&& 1 \\

    \hline
    \multicolumn{4}{|c|}{\textbf{Изоляция немобильности}} \\

    \hline
    Г0901 & Оценка локализации непереносимой части программы & » & 1 \\

    \hline
    &&& 1 \\

    \hline
    \multicolumn{4}{|c|}{\textbf{Простота кодирования}} \\

    \hline
    Г1001& Оценка использования отрицательных или булевых выражений & » & 1 \\

    \hline
    Г1002 & Оценка программы по использованию условных переходов & » & 1 \\

    \hline
    Г1003& Оценка программы по использованию безусловных переходов  & » & 0 \\

    \hline
    Г1004 & Оформление процедур входа и выхода из циклов  & » & 1 \\

    \hline
    Г1005 & Ограничения на модификацию переменной индексации в цикле  & » & 1 \\

    \hline
    Г1007 & Оценка программы по использованию локальных переменных  & » & 1 \\

    \hline
    Г1006 & Оценка модулей по направлению потока управления & » & 0 \\

    \hline
    &&& 0,7 \\

    \hline
    \multicolumn{4}{|c|}{\textbf{Число комментариев}} \\

    \hline
    Г1101 & Оценка программы по числу комментариев & » & 1 \\

    \hline
    &&& 1 \\

    \hline
    \multicolumn{4}{|c|}{\textbf{Качество комментариев}} \\

    \hline
    Г1201 & Наличие заголовка в программе  & » & 1 \\

    \hline
    Г1202& Комментарии к точкам ветвлений  & » & 1 \\

    \hline
    Г1203 & Комментарии к машинозависимым частям программы & » & 1 \\

    \hline
    Г1204 & Комментарии к машинозависимым операторам программы  & » & 0 \\

    \hline
    Г1205 & Комментарии к операторам объявления переменных  & » & 1 \\

    \hline
    Г1206 & Оценка семантики операторов  & » & 1 \\

    \hline
    Г1207 & Наличие соглашений по форме представления комментариев  & » & 0 \\

    \hline
    Г1208 & Наличие общих комментариев к программам & » & 1 \\

    \hline
    &&& 0,7 \\

    \hline
    \multicolumn{4}{|c|}{\textbf{Использование описательных средств языка}} \\

    \hline
    Г1301 & Использование языков высокого уровня & » & 1 \\

    \hline
    Г1302& Семантика имен используемых переменных& » & 1 \\

    \hline
    Г1303 & Использование отступов, сдвигов и пропусков при формировании текста & » & 1 \\

    \hline
    Г1304& Размещение операторов по строкам & » & 1 \\

    \hline
    &&& 1 \\

    \hline
    \multicolumn{4}{|c|}{\textbf{Независимость модулей}} \\


    \hline
    Г1401 & Передача информации для управления по параметрам & » & 1 \\

    \hline
    Г1402 & Наличие передачи результатов работы между модулями & » & 1 \\

    \hline
    Г1403 & Наличие проверки правильности данных, получаемых модулями от вызываемого модуля & » & 1 \\

    \hline
    Г1404 & Использование общих областей памяти & » & 1 \\

    \hline
    Г1405 & Параметрическая передача входных данных & » & 1 \\


    \hline
    &&& 1 \\


    \hline
\end{ztable}

\textbf{Показатели корректности}

\begin{ztable}{|p{2cm}|p{10cm}|p{2cm}|p{2cm}|}{Оценочные элементы фактора “Корректность ПС”}{correctness_table}
    \hline
    Код элемента & Наименование & Метод оценки & Оценка\\

    \endhead

    \hline
    \multicolumn{4}{|c|}{ \textbf{Требования, предъявляемые к полноте документации разработчика}} \\

    \hline
    К0101 & Наличие всех необходимых документов для понимания и использования ПС & Экспертный & 1 \\

    \hline
    К0102 & Наличие описания и схемы иерархии модулей программы & » & 1 \\

    \hline
    К0103& Наличие описания основных функций & » & 1 \\

    \hline
    К0104 & Наличие описания частных функций & » & 1 \\

    \hline
    К0105 & Наличие описания данных & » & 1 \\

    \hline
    К0106 & Наличие описания алгоритмов & » & 1 \\

    \hline
    К0107 & Наличие описания интерфейсов между модулями & » & 1 \\

    \hline
    К0108 & Наличие описания интерфейсов  с пользователями & » & 1 \\

    \hline
    К0109 & Наличие описания используемых числовых методов & » & 0 \\

    \hline
    К0110& Указаны ли все численные методы & » & 0 \\

    \hline
    К0111& Наличие описания всех параметров & » & 0 \\

    \hline
    К0112 & Наличие описания методов настройки системы & » & 1 \\

    \hline
    К0113 & Наличие описания всех  диагностических сообщений & » & 1 \\

    \hline
    К0114 & Наличие описания способов проверки работоспособности программы & » & 1 \\

    \hline
    &&& 0,8 \\



    \hline
    \multicolumn{4}{|c|}{ \textbf{Полнота программной документации}} \\

    \hline
    К0201 & Реализация всех исходных модулей & » & 1 \\

    \hline
    К0202 & Реализация всех основных функций & » & 1 \\

    \hline
    К0203 & Реализация всех частных  функций & » & 1 \\

    \hline
    К0204 & Реализация всех алгоритмов & » & 1 \\

    \hline
    К0205 & Реализация всех взаимосвязей в системе & » & 1 \\

    \hline
    К0206 & Реализация всех интерфейсов между модулями & » & 1 \\

    \hline
    К0207 & Реализация возможности настройки системы & » & 1 \\

    \hline
    К0208 & Реализация диагностики всех граничных и аварийных ситуаций & » & 1 \\

    \hline
    К0209 & Наличие определения всех данных (переменные, индексы, массивы и проч.) & » & 1 \\

    \hline
    К0210 & Наличие интерфейсов с пользователем & » & 1 \\

    \hline
    &&& 1 \\



    \hline
    \multicolumn{4}{|c|}{ \textbf{Непротиворечивость документации разработчика}} \\


    \hline
    К0301  & Отсутствие противоречий в описании частных функций & » & 1 \\

    \hline
    К0302 & Отсутствие противоречий в описании основных функций в разных документах & » & 1 \\

    \hline
    К0303 & Отсутствие противоречий в описании алгоритмов & » & 1 \\

    \hline
    К0304  & Отсутствие противоречий в описании взаимосвязей в системе & » & 1 \\

    \hline
    К0305 & Отсутствие противоречий в описании интерфейсов между модулями & » & 1 \\

    \hline
    К0306  & Отсутствие противоречий в описании интерфейсов с пользователем & » & 1 \\

    \hline
    К0307  & Отсутствие противоречий в описании настройки системы & » & 1 \\

    \hline
    К0309  & Отсутствие противоречий в описании иерархической структуры сообщений & » & 1 \\

    \hline
    К0310 & Отсутствие противоречий в описании диагностических сообщений & » & 1 \\

    \hline
    К0311 & Отсутствие противоречий в описании данных & » & 1 \\


    \hline
    &&& 1 \\



    \hline
    \multicolumn{4}{|c|}{ \textbf{Непротиворечивость программы}} \\

    \hline
    К0401 & Отсутствие противоречий в выполнении основных функций & » & 1 \\

    \hline
    К0402 & Отсутствие противоречий в выполнении частных функций & » & 1 \\

    \hline
    К0403 & Отсутствие противоречий в выполнении алгоритмов & » & 1 \\

    \hline
    К0404 & Правильность взаимосвязей & » & 1 \\

    \hline
    К0405 & Правильность реализации интерфейса между модулями & » & 1 \\

    \hline
    К0406 & Правильность реализации интерфейса с пользователем & » & 1 \\

    \hline
    К0407 & Отсутствие противоречий в настройке системы & » & 1 \\

    \hline
    К0408 & Отсутствие противоречий в диагностике системы & » & 1 \\

    \hline
    К0409 & Отсутствие противоречий в общих переменных & » & 1 \\

    \hline
    &&& 1 \\



    \hline
    \multicolumn{4}{|c|}{ \textbf{Единообразие интерфейсов между модулями и пользователями}} \\

    \hline
    К0501 & Единообразие способов вызова модулей & » & 1 \\

    \hline
    К0502 & Единообразие процедур возврата управления из модулей & » & 1 \\

    \hline
    К0503 & Единообразие способов сохранения информации для возврата & » & 0 \\

    \hline
    К0504  & Единообразие способов восстановления информации для возврата & » & 0 \\

    \hline
    К0505  & Единообразие организации списков передаваемых параметров & » & 0 \\

    \hline
    &&& 0,4 \\



    \hline
    \multicolumn{4}{|c|}{ \textbf{Единообразие кодирования и определения переменных}} \\

    \hline
    К0601 & Единообразие наименования каждой переменной и константы & » & 1 \\

    \hline
    К0602 & Все ли одинаковые константы встречаются во всех программах под одинаковыми именами & » & 0 \\

    \hline
    К0603 & Единообразие определения внешних данных во всех программах & » & 1 \\

    \hline
    К0604 & Используются ли разные идентификаторы для разных переменных & » & 1 \\

    \hline
    К0605 & Все ли общие переменные объявлены как общие переменные & » & 1 \\

    \hline
    К0606 & Наличие определений одинаковых атрибутов & » & 1 \\

    \hline
    &&& 0,8 \\



    \hline
    \multicolumn{4}{|c|}{ \textbf{Соответствие документации стандартам}} \\

    \hline
    К0701 & Комплектность документации в соответствии со стандартами & » & 1 \\

    \hline
    К0702 & Правильное оформление частей документов & » & 1 \\

    \hline
    К0703 & Правильное оформление титульных и заглавных листов документов & » & 1 \\

    \hline
    К0704 & Наличие в документах всех разделов в соответствии со стандартами & » & 1 \\

    \hline
    К0705 & Полнота содержания разделов в соответствии со стандартами & » & 0 \\

    \hline
    К0706 & Деление документов на структурные элементы: разделы, подразделы, пункты, подпункты & » & 1 \\

    \hline
    &&& 0,8 \\



    \hline
    \multicolumn{4}{|c|}{ \textbf{Соответствие ПС стандартам программирования}} \\

    \hline
    К0801 & Соответствие организации и вычислительного процесса эксплуатационной документации & » & 1 \\

    \hline
    К0802 & Правильность заданий на выполнение программы, правильность написания управляющих и операторов (отсутствие ошибок) & » & 1 \\

    \hline
    К0803& Отсутствие ошибок в описании действий пользователя & » & 1 \\

    \hline
    К0804 & Отсутствие ошибок в описании запуска & » & 1 \\

    \hline
    К0805 & Отсутствие ошибок в описании генерации & » & 1 \\

    \hline
    К0806 & Отсутствие ошибок в описании настройки & » & 1 \\

    \hline
    &&& 1 \\



    \hline
    \multicolumn{4}{|c|}{ \textbf{Полнота тестирования проекта}} \\

    \hline
    К1001 & Наличие требований к тестированию программ & » & 0 \\

    \hline
    К1002 & Достаточность требований к тестированию программ & » & 0 \\

    \hline
    К1003 & Отношение числа модулей, отработавших в процессе тестирования и отладки (Qтм) к общему числу модулей (Qтм) & Расчетный & 1 \\

    \hline
    К1004 & Отношение числа логических блоков, отработавших в процессе тестирования и отладки (Qтб), к общему числу логических блоков в программе (Qтб) & То же & 1 \\

    \hline
    &&& 0,7 \\



    \hline
\end{ztable}


Абсолютные показатели критериев i-ого фактора качества определяется по формуле:
$$P_i = \sum_{k=0}^{n}(P^M_{jk} * V^M_{jk})$$
, где

$P^M_{jk}$ - итоговая оценка k-той метрики j-того критерия;

$V^M_{jk}$ - весовой коэффициент j-того показателя;

$n$ - число метрик, относящихся к j-тому критерию.

Таким образом, абсолютные показатели составляют:

\begin{ztable}{|p{5cm}|p{5cm}|}{ Результаты оценки качества программного продукта} {quality_result_table}
  \hline
  Фактор качества & Оценка\\

  \endhead

  \hline
  Надёжность & 0,7 \\
  \hline
  Сопровождаемость & 0,5 \\
  \hline
  Удобство применения & 0,9 \\
  \hline
  Эффективность & 0,9 \\
  \hline
  Универсальность & 0,7 \\
  \hline
  Корректность & 0,7 \\
  \hline
\end{ztable}

Все показатели принимают значения в пределах требуемой нормы.

\textbf{Выводы}

В результате проделанной работы была произведена оценка качества программного продукта
“Программа для расчета и предупреждения пересечения стволов нефтяных скважин”.

Показатель оценки надёжности равен 0,7. Эта величина показывает, что программа оснащена определенными базовыми
методами защиты от сбоев и злоумышленников.

Высокий показатель универсальности равен 0,7 говорит о том, что данный модуль может быть перенесен в другие приложения.

Значение показателя сопровождаемости равное 0,5 говорит о необходимости дальнейшей работы по улучшению наглядности и
устойчивости функционирования.

Полученные оценки 0,9 означают, что программа достаточно эффективна и удобна в применении.


\subsection{Вычислительный эксперимент и анализ результатов}

??? - тут вот тоже нифига не понятно. Может убрать главу совсем. Никто не заметит=)

\newpage
\section*{ЗАКЛЮЧЕНИЕ}
\addcontentsline{toc} {section} {ЗАКЛЮЧЕНИЕ}

\subsection*{Результаты работы}
\addcontentsline{toc} {subsection} {Результаты работы}

В представленной работе описан программный продукт для предотвращения пересечения стволов скважин.
Он позволяет осуществлять следующие действия:
импортировать данные измерительных устройств, визуализировать траектории стволов и оценивать расстояния между ними.

Такой программный продукт может найти применение в строительстве с использованием технологии искусственного замораживания
грунтов, а так же в бурении нефтяных и газовых скважин.

Для работы с данной программой требуется знания в области нефтяной промышленности, а также понимание принципов инклинометрии.
Требования к наличию знаний компьютерных навыков минимальна.

\subsection*{Выводы}
\addcontentsline{toc} {subsection} {Выводы}
% В результате разработки было получено ПО, которое решает пост

\newpage
\section*{СПИСОК ИСПОЛЬЗОВАННЫХ ИСТОЧНИКОВ}
\addcontentsline{toc} {section} {СПИСОК ИСПОЛЬЗОВАННЫХ ИСТОЧНИКОВ}

\begin{thebibliography}{99}
  \bibitem{wiki} Web-портал http://ru.wikipedia.org/ - свободная энциклопедия.
  \bibitem{gost} ГОСТ 28195-89 – «Оценка качества программных средств».
  \bibitem{qt} Макс Шлее Qt 4.8. Профессиональное программирование на C++, — СПб.: БХВ-Петербург, 2012. — 894 стр.
  % \bibitem{metro} Портал http://www.metro.ru/library/stroitelstvo_metropolitenov/512/ - московское метро.
  % \bibitem{mountain} Портал http://www.mining-enc.ru/z/zamorazhivanie-gruntov/ - горная энциклопедия.
  % \bibitem{uhta} http://lib.ugtu.net/sites/default/files/books/2014/bliznyukov_v.yu_._raschyot_i_korrektirovanie_traektorii_skvazhiny_pri_burenii_2014.pdf
\end{thebibliography}

\newpage
\section*{ПРИЛОЖЕНИЯ}
\addcontentsline{toc}{section}{ПРИЛОЖЕНИЯ}

\subsection*{Программная документация}
\addcontentsline{toc}{subsection}{Программная документация}

\subsubsection*{Техническое задание}
\addcontentsline{toc}{subsection}{Техническое задание}

\subsubsection*{Руководство программиста}
\addcontentsline{toc}{subsection}{Руководство программиста}

\subsubsection*{Руководство пользователя}
\addcontentsline{toc}{subsection}{Руководство пользователя}

\end{document}
