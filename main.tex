\documentclass[a4paper,14pt]{report} %размер бумаги устанавливаем А4, шрифт 12пунктов
\usepackage[T2A]{fontenc}
\usepackage[utf8]{inputenc}%включаем свою кодировку: koi8-r или utf8 в UNIX, cp1251 в Windows
\usepackage[english,russian]{babel}%используем русский и английский языки с переносами
\usepackage{amssymb,amsfonts,amsmath,mathtext,cite,enumerate,float} %подключаем нужные пакеты расширений
\usepackage[dvips]{graphicx} %хотим вставлять в диплом рисунки?
\graphicspath{{images/}}%путь к рисункам

\makeatletter
\renewcommand{\@biblabel}[1]{#1.} % Заменяем библиографию с квадратных скобок на точку:
\makeatother

\usepackage{geometry} % Меняем поля страницы
\geometry{left=3cm}% левое поле
\geometry{right=1cm}% правое поле
\geometry{top=2cm}% верхнее поле
\geometry{bottom=2cm}% нижнее поле

\renewcommand{\theenumi}{\arabic{enumi}}% Меняем везде перечисления на цифра.цифра
\renewcommand{\labelenumi}{\arabic{enumi}}% Меняем везде перечисления на цифра.цифра
\renewcommand{\theenumii}{.\arabic{enumii}}% Меняем везде перечисления на цифра.цифра
\renewcommand{\labelenumii}{\arabic{enumi}.\arabic{enumii}.}% Меняем везде перечисления на цифра.цифра
\renewcommand{\theenumiii}{.\arabic{enumiii}}% Меняем везде перечисления на цифра.цифра
\renewcommand{\labelenumiii}{\arabic{enumi}.\arabic{enumii}.\arabic{enumiii}.}% Меняем везде перечисления на цифра.цифра

\begin{document}
\begin{titlepage}
\newpage

\begin{center}
Федеральное государственное бюджетное образовательное учреждение\\
высшего профессионального образования \\
\vspace{0.5cm}
УФИМСКИЙ ГОСУДАРСТВЕННЫЙ АВИАЦИОННЫЙ ТЕХНИЧЕСКИЙ УНИВЕРСИТЕТ\\
\vspace{0.5cm}
ФАКУЛЬТЕТ  ИНФОРМАТИКИ  И  РОБОТОТЕХНИКИ \\
\vspace{0.5cm}
КАФЕДРА  ВЫЧИСЛИТЕЛЬНОЙ  МАТЕМАТИКИ  И  КИБЕРНЕТИКИ\\
\vspace{0.5cm}
Направление 231000 – Программная инженерия
\end{center}

\vspace{2em}

\begin{center}
\Large {ВЫПУСКНАЯ КВАЛИФИКАЦИОННАЯ РАБОТА}

\vspace{2.5em}

Тема: Программа для расчета и предупреждения пересечения стволов нефтяных скважин
\end{center}

\vspace{3em}
\noindent
\begin{tabularx}{\textwidth}{|l|c|X|X|}
\hline
& ФИО & Подпись & Дата\\
\hline
Студент & Синявский Г. Н. &&\\
\hline
Руководитель работы & Еникеева К. Р. &&\\
\hline
Консультант & Еникеева К. Р. &&\\
\hline
Контроль программного продукта &&&\\
\hline
Председатель комиссии по предзащите &&&\\
\hline
Рецензент &&& \\
\hline
\end{tabularx}

\vspace{2em}
\center Допущен к защите\\
Зав. кафедрой  ВМК, д.т.н., проф.\\
\underline{\hspace{5cm}} Н.И. Юсупова\\
“\underline{\hspace{1cm}}”\underline{\hspace{5cm}} 2015 г.

\vspace{\fill}

\begin{center}
УФА~--- 2015г.
\end{center}

\end{titlepage}
% это титульный лист
\tableofcontents % это оглавление, которое генерируется автоматически
\end{document}