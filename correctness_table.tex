\textbf{Показатели корректности}

\begin{ztable}{|p{2cm}|p{10cm}|p{2cm}|p{2cm}|}{Оценочные элементы фактора “Корректность ПС”}{correctness_table}
    \hline
    Код элемента & Наименование & Метод оценки & Оценка\\

    \endhead

    \hline
    \multicolumn{4}{|c|}{ \textbf{Требования, предъявляемые к полноте документации разработчика}} \\

    \hline
    К0101 & Наличие всех необходимых документов для понимания и использования ПС & Экспертный & 1 \\

    \hline
    К0102 & Наличие описания и схемы иерархии модулей программы & » & 1 \\

    \hline
    К0103& Наличие описания основных функций & » & 1 \\

    \hline
    К0104 & Наличие описания частных функций & » & 1 \\

    \hline
    К0105 & Наличие описания данных & » & 1 \\

    \hline
    К0106 & Наличие описания алгоритмов & » & 1 \\

    \hline
    К0107 & Наличие описания интерфейсов между модулями & » & 1 \\

    \hline
    К0108 & Наличие описания интерфейсов  с пользователями & » & 1 \\

    \hline
    К0109 & Наличие описания используемых числовых методов & » & 0 \\

    \hline
    К0110& Указаны ли все численные методы & » & 0 \\

    \hline
    К0111& Наличие описания всех параметров & » & 0 \\

    \hline
    К0112 & Наличие описания методов настройки системы & » & 1 \\

    \hline
    К0113 & Наличие описания всех  диагностических сообщений & » & 1 \\

    \hline
    К0114 & Наличие описания способов проверки работоспособности программы & » & 1 \\

    \hline
    &&& 0,8 \\



    \hline
    \multicolumn{4}{|c|}{ \textbf{Полнота программной документации}} \\

    \hline
    К0201 & Реализация всех исходных модулей & » & 1 \\

    \hline
    К0202 & Реализация всех основных функций & » & 1 \\

    \hline
    К0203 & Реализация всех частных  функций & » & 1 \\

    \hline
    К0204 & Реализация всех алгоритмов & » & 1 \\

    \hline
    К0205 & Реализация всех взаимосвязей в системе & » & 1 \\

    \hline
    К0206 & Реализация всех интерфейсов между модулями & » & 1 \\

    \hline
    К0207 & Реализация возможности настройки системы & » & 1 \\

    \hline
    К0208 & Реализация диагностики всех граничных и аварийных ситуаций & » & 1 \\

    \hline
    К0209 & Наличие определения всех данных (переменные, индексы, массивы и проч.) & » & 1 \\

    \hline
    К0210 & Наличие интерфейсов с пользователем & » & 1 \\

    \hline
    &&& 1 \\



    \hline
    \multicolumn{4}{|c|}{ \textbf{Непротиворечивость документации разработчика}} \\


    \hline
    К0301  & Отсутствие противоречий в описании частных функций & » & 1 \\

    \hline
    К0302 & Отсутствие противоречий в описании основных функций в разных документах & » & 1 \\

    \hline
    К0303 & Отсутствие противоречий в описании алгоритмов & » & 1 \\

    \hline
    К0304  & Отсутствие противоречий в описании взаимосвязей в системе & » & 1 \\

    \hline
    К0305 & Отсутствие противоречий в описании интерфейсов между модулями & » & 1 \\

    \hline
    К0306  & Отсутствие противоречий в описании интерфейсов с пользователем & » & 1 \\

    \hline
    К0307  & Отсутствие противоречий в описании настройки системы & » & 1 \\

    \hline
    К0309  & Отсутствие противоречий в описании иерархической структуры сообщений & » & 1 \\

    \hline
    К0310 & Отсутствие противоречий в описании диагностических сообщений & » & 1 \\

    \hline
    К0311 & Отсутствие противоречий в описании данных & » & 1 \\


    \hline
    &&& 1 \\



    \hline
    \multicolumn{4}{|c|}{ \textbf{Непротиворечивость программы}} \\

    \hline
    К0401 & Отсутствие противоречий в выполнении основных функций & » & 1 \\

    \hline
    К0402 & Отсутствие противоречий в выполнении частных функций & » & 1 \\

    \hline
    К0403 & Отсутствие противоречий в выполнении алгоритмов & » & 1 \\

    \hline
    К0404 & Правильность взаимосвязей & » & 1 \\

    \hline
    К0405 & Правильность реализации интерфейса между модулями & » & 1 \\

    \hline
    К0406 & Правильность реализации интерфейса с пользователем & » & 1 \\

    \hline
    К0407 & Отсутствие противоречий в настройке системы & » & 1 \\

    \hline
    К0408 & Отсутствие противоречий в диагностике системы & » & 1 \\

    \hline
    К0409 & Отсутствие противоречий в общих переменных & » & 1 \\

    \hline
    &&& 1 \\



    \hline
    \multicolumn{4}{|c|}{ \textbf{Единообразие интерфейсов между модулями и пользователями}} \\

    \hline
    К0501 & Единообразие способов вызова модулей & » & 1 \\

    \hline
    К0502 & Единообразие процедур возврата управления из модулей & » & 1 \\

    \hline
    К0503 & Единообразие способов сохранения информации для возврата & » & 0 \\

    \hline
    К0504  & Единообразие способов восстановления информации для возврата & » & 0 \\

    \hline
    К0505  & Единообразие организации списков передаваемых параметров & » & 0 \\

    \hline
    &&& 0,4 \\



    \hline
    \multicolumn{4}{|c|}{ \textbf{Единообразие кодирования и определения переменных}} \\

    \hline
    К0601 & Единообразие наименования каждой переменной и константы & » & 1 \\

    \hline
    К0602 & Все ли одинаковые константы встречаются во всех программах под одинаковыми именами & » & 0 \\

    \hline
    К0603 & Единообразие определения внешних данных во всех программах & » & 1 \\

    \hline
    К0604 & Используются ли разные идентификаторы для разных переменных & » & 1 \\

    \hline
    К0605 & Все ли общие переменные объявлены как общие переменные & » & 1 \\

    \hline
    К0606 & Наличие определений одинаковых атрибутов & » & 1 \\

    \hline
    &&& 0,8 \\



    \hline
    \multicolumn{4}{|c|}{ \textbf{Соответствие документации стандартам}} \\

    \hline
    К0701 & Комплектность документации в соответствии со стандартами & » & 1 \\

    \hline
    К0702 & Правильное оформление частей документов & » & 1 \\

    \hline
    К0703 & Правильное оформление титульных и заглавных листов документов & » & 1 \\

    \hline
    К0704 & Наличие в документах всех разделов в соответствии со стандартами & » & 1 \\

    \hline
    К0705 & Полнота содержания разделов в соответствии со стандартами & » & 0 \\

    \hline
    К0706 & Деление документов на структурные элементы: разделы, подразделы, пункты, подпункты & » & 1 \\

    \hline
    &&& 0,8 \\



    \hline
    \multicolumn{4}{|c|}{ \textbf{Соответствие ПС стандартам программирования}} \\

    \hline
    К0801 & Соответствие организации и вычислительного процесса эксплуатационной документации & » & 1 \\

    \hline
    К0802 & Правильность заданий на выполнение программы, правильность написания управляющих и операторов (отсутствие ошибок) & » & 1 \\

    \hline
    К0803& Отсутствие ошибок в описании действий пользователя & » & 1 \\

    \hline
    К0804 & Отсутствие ошибок в описании запуска & » & 1 \\

    \hline
    К0805 & Отсутствие ошибок в описании генерации & » & 1 \\

    \hline
    К0806 & Отсутствие ошибок в описании настройки & » & 1 \\

    \hline
    &&& 1 \\



    \hline
    \multicolumn{4}{|c|}{ \textbf{Полнота тестирования проекта}} \\

    \hline
    К1001 & Наличие требований к тестированию программ & » & 0 \\

    \hline
    К1002 & Достаточность требований к тестированию программ & » & 0 \\

    \hline
    К1003 & Отношение числа модулей, отработавших в процессе тестирования и отладки (Qтм) к общему числу модулей (Qтм) & Расчетный & 1 \\

    \hline
    К1004 & Отношение числа логических блоков, отработавших в процессе тестирования и отладки (Qтб), к общему числу логических блоков в программе (Qтб) & То же & 1 \\

    \hline
    &&& 0,7 \\



    \hline
\end{ztable}
