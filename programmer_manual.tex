\textbf{Назначение и условия применения программы}

Приложение для предупреждения пересечения стволов скважин.
Программный продукт должен работать на любых ПК с x86-совместимым процессором с частотой 1Ггц и выше,
оперативной памятью не менее 512мб и доступным дисковым пространством минимум 200Мб, работающий под
управлением ОС Windows 7 или ОС на базе ядра Linux.

\textbf{Структура программы}

Программа написана с использованием архитектурного подхода MVC и состоит из набора классов. Один cpp файл содержит только
один класс, каждый cpp файл имеет соответствующий одноименный h файл. Файлы исходных кодов сгруппированы в следующие
поддиректории:
\begin{itemize}
  \item корень проекта - содержит классы основных окон, виджеты для визуализации данных и некоторые вспомогательные классы;
  \item delegates - т.н. делегаты, классы, отвечающие за отображение данных в ячейках таблиц и списков;
  \item dialogs - классы, отвечающие за логику работы диалоговых окон;
  \item entities - классы, описывающие базовые структуры данных, вроде Подрядчика или Месторождения;
  \item import\_wizard - классы, отвечающие за логику мастера импорта данных;
  \item log - классы, отвечающие за логику журналов;
  \item menus - классы, отвечающие за различные контекстные меню;
  \item mixins - вспомогательные классы, от которых наследуются некоторые классы приложения;
  \item models - модели данных, большая часть из них описывает таблицы БД;
  \item views - классы, отвечающие за отображение моделей;
\end{itemize}

Программа хранит свои настройки с использованием абстракции над стандартной системой хранения настроек для текущей платформы:
для Windows это реестр, для Linux-систем - это текстовый файл ~/.config/SPT/Collisions.conf.

БД продукта представляет собой файл db.sqlite, он может быть прочитан и отредактирован любой, поддерживающей формат sqlite, утилитой.

\textbf{Сообщения программисту}

В программе не предусмотрен вывод сообщений специально для программиста, однако в ходе работы программы могут появиться
общие сообщения программы.
