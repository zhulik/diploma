\newpage
\section*{Введение}
\addcontentsline{toc} {section} {Введение}

При бурении близкорасположенных скважин очень остро стоит проблема предотвращения пересечения стволов скважин. Связано это, в первую очередь с финансовыми и временными
расходами на ликвидацию последствий такой аварии: герметизация пересекаемой скважины, ликвидация участка пересекшей и восстановление траектории пересекаемой скважин.
При бурении с кустовых площадок риск пересечения стволов минимизируется бурением скважин, отдаляющихся друг от друга.
Такой способ предотвращения пересечения стволов не подходит при бурении замораживающих скважин, которые должные располагаться на строго определенном расстоянии
друг от друга.


\subsection*{Описание предметной области}
\addcontentsline{toc} {subsection} {Описание предметной области}

Замораживание грунтов — искусственное охлаждение грунтов в естественном залегании до отрицательных температур с целью их упрочения и достижения необходимой степени водонепроницаемости.

Замораживание грунтов возможно при различных глубинах, сочетаниях грунтов, скоростях движения грунтовых вод и степени их минерализации. Замораживание грунтов —
основной способ при работе в сложных гидрогеологических условиях как при замораживании водоносных рыхлых, так и водоносных трещиноватых пород.

Замораживание грунтов применяют при возведении фундаментов зданий и сооружений, строительстве шахт, тоннелей, метрополитенов, противофильтрационных завес, мостов, плотин, подземных хранилищ и др.

Для охлаждения грунта используют холодильные установки с системой погружаемых в грунт труб (замораживающих колонок), по которым циркулирует холодоноситель,
охлаждённый до -20 -40°С (рассольный способ замораживания), или хладагент, который непосредственно испаряется в замораживающей колонке при температуре от -35 до -196°С
(безрассольный способ замораживания). В качестве холодоносителя применяют водные растворы солей (например, хлориды кальция, натрия, лития) или специальные жидкости,
которые замерзают при низких температураx, а в качестве хладагента — аммиак, углекислоту, фреон и др. В процессе непрерывного теплообмена холодоносителя (хладагента)
с грунтом вокруг каждой трубы образуются ледопородные цилиндры, которые в дальнейшем смыкаются, образуя замкнутое ледопородное ограждение по контуру подземного сооружения
или массив замороженного грунта.

Для предупреждения протекания ледопородного ограждения необходимо строго контролировать расстояния между замораживающими скважинами не допускать их пересечения.


\subsection*{Мотивация, актуальность проблемы}
\addcontentsline{toc} {subsection} {Мотивация, актуальность проблемы}

Решения о написании собственного продукта связано со сложностью и дороговизной существующих решений
% ??? - жду от заказчика названия пакетов, кое-какие нашел сам, но цены там только по требованию и сильно зависят от покупателя

\begin{ztable}{|p{5cm}|p{5cm}|p{5cm}|}{Существующие решения}{solutions_table}
    \hline
    Название продукта & Недостатки & Стоимость\\

    \endhead


    \hline
    Landmark Compass &
    \begin{itemize}
      \item Стоимость
      \item Необходимость обучения сотрудников
      \item Сложность внедрения и поддержки
      \item Избыточный функционал
    \end{itemize}
    & Порядка \$170000
    \\

    \hline
    Paradigm Sysdrill &
    \begin{itemize}
      \item Стоимость
      \item Необходимость обучения сотрудников
      \item Избыточный функционал
    \end{itemize}
    & Порядка \$150000
    \\

    \hline
\end{ztable}


\subsection*{Цели, задачи ВКР}
\addcontentsline{toc} {subsection} {Цели, задачи ВКР}
Целью дипломной работы является разработка программного обеспечения, позволяющего визуализировать, усреднять и производить анализ замеров стволов скважин,
на основании данных, полученных с измерительного оборудования. Для достижения поставленной цели необходимо решить следующие задачи:
\begin{itemize}
  \item провести анализ существующих программных продуктов;
  \item разработка функциональной и информационной моделей, программного обеспечения;
  \item разработка модуля импорта данных
  \item разработка системы управления содержимым БД и усреднением замеров
  \item разработка модуля визуализации замеров
  \item разработка модуля расчетов расстояний между стволами
  \item разработка модуля визуализации расстояний между стволами
\end{itemize}

\subsection*{Содержание работы по главам}
\addcontentsline{toc} {subsection} {Содержание работы по главам}

???
