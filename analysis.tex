\newpage
\section{Анализ проблемы и постановка задачи}
\subsection{Анализ предметной области}
Бурение скважин — это процесс сооружения направленной цилиндрической горной выработки в земле, диаметр "D" которой мал по сравнению с её длиной по стволу "H",
без доступа человека на забой. Начало скважины на поверхности земли называют устьем, дно — забоем, а стенки скважины образуют ее ствол.

Кустовое бурение — сооружение скважин (в основном наклонно направленных), устья которых группируются на близком расстоянии друг от друга на общей ограниченной
площадке (основании). Применяется при разработке месторождений под застроенными участками, при разработке нефтяных и газовых месторождений в определённых
климатических условиях (например, в зимний период, когда наблюдается большой снеговой покров, или весной во время распутицы и значительных паводков),
на территории с сильно пересечённым рельефом местности или в пределах акваторий.

В настоящее время практически все эксплуатационные скважины бурятся кустовым методом, когда устья нескольких скважин в кусте расположены близко друг к другу (4–5
м) на одной технологической площадке, а забои находятся в узлах сетки разработки. Число скважин в кусте колеблется от  2 до нескольких десятков.

Искусственное искривление скважин применяется с целью:
\begin{itemize}
  \item добычи нефти и газа из труднодоступных участков, занятых на поверхности промышленными и жилыми объектами, оврагами, горами, реками, озерами, болотами, лесами, морями;
  \item экономии отводимых под строительство скважин плодородных земельных участков, лесов и др.;
  \item экономии затрат на строительство оснований, подъездных путей, линий электропередач, связи, трубопроводов;
  \item сокращения средств и времени на строительно-монтажные работы и обслуживание при эксплуатации скважин с близко расположенными устьями;
  \item обхода зон катастрофических поглощений, обвалов и аварий в стволе скважины;
  \item вскрытия продуктивных пластов, залегающих под пологим сбросом или между двумя параллельными сбросами;
  \item проходки стволов на нефтяные пласты, залегающие под соляными куполами, в связи с трудностью бурения через них (соль «плывет», срезает бурильные и обсадные колонны);
  \item бурения стволов для глушения открытых фонтанов и тушения пожаров;
  \item перебуривания части ствола скважины;
  \item вскрытия продуктивного пласта под определенным углом для увеличения поверхности дренажа и дебита скважины;
  \item многозабойного вскрытия продуктивного пласта.
\end{itemize}

Помимо искусственного искривления скважин так же имеет место быть самопроизвольное искривление, связанное с геологическими, техническими и технологическими факторами.

\textbf{Геологические факторы:}
\begin{itemize}
  \item перемежаемость по твердости - чередование мягких и твердых горных пород;
  \item жеоды - инородные тела в составе горное породы, отличающиеся по твердости;
\end{itemize}

\textbf{Технические факторы:}
\begin{itemize}
  \item несоосность вышки  относительно осей стола ротора  и шахтового направления;
  \item негоризонтальность стола ротора;
  \item использование искривленных труб (ведущих и бурильных) и труб, у которых резьбы нарезаны под углом;
  \item эксцентричное забуривание нижележащего участка скважины.
\end{itemize}

\textbf{Технологические факторы:}
\begin{itemize}
  \item влияние осевой нагрузки;
  \item влияние частоты вращения бурильной колонны.
\end{itemize}

\textbf{Методы предупреждения самопроизвольного искривления скважин}
\begin{itemize}
  \item для предупреждения естественного искривления скважин необходимо исключить или уменьшить действие управляемых технических факторов и нейтрализовать действие неуправляемых геологических условий;
  \item технические причины искривлений должны быть устранены до начала бурения скважины;
  \item действие технологических причин искривления могут быть сведены к минимуму центрированием низа бурильной колонны, увеличением его жесткости, регулированием осевой нагрузки;
  \item цель центрирования нижней части бурильной колонны - препятствовать отклонению оси долота от оси скважины;
  \item увеличение жесткости и массы нижней части бурильной колонны повышает устойчивость к изгибу, уменьшает длину сжатой части, позволяет использовать повышенные нагрузки на долото;
  \item для компенсации геологических причин искривления (наклонно-залегающие анизотропные породы) можно использовать методы наклонно направленного разбуривания ствола в направлении, противоположном естественному искривлению;
  \item метод буровых трасс – перенос устья скважины по азимуту и величине смещения при самопроизвольном искривлении скважин.
\end{itemize}

В связи с неуправляемым действием геологических условий, при бурении следует контролировать данные инклинометрии и вовремя принимать решения об искусственном искривлении
ствола для предотвращения пересечения скважин и минимизации отхода траектории от проектной.

Поскольку глубина промерзания грунта вокруг ствола замораживающей скважины равна примерно 70см, то, с учетом допуска и погрешности, такие скважины должны буриться
на расстоянии примерно 119см друг от друга по всей длине ствола, что требует дополнительного контроля за данным инклинометрии и увеличивает вероятность их пересечения.

В настоящей работе описан программный продукт, позволяющий анализировать данные инклинометрии и оценивать расстояния между стволами по всей их длине, тем
самым упрощая принятие решения об искусственном искривлении стволов и предупреждении их пересечения.




\subsection{Содержательная постановка проблемы}

\subsection{Формальная постановка задачи}
Формальной постановке задачи соответствует контекстная диаграмма методологии IDEF0, описывающая входные и выходные данные, управляющие воздействия и механизмы,
влияющие на систему в целом, приведённая на рисунке 1.1.:
%TODO: Добавить картинку и проверить нумерацию

\subsection{Структура решения задачи, декомпозиция задачи на подзадачи}
