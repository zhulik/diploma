\newpage
\section{Анализ проблемы и постановка задачи}



\subsection{Анализ предметной области}

\subsection{Содержательная постановка проблемы}

\subsection{Формальная постановка задачи}
\subsection{Структура решения задачи, декомпозиция задачи на подзадачи}
Целью дипломной работы является разработка программиного обеспечения, позволяющего визуализировать, усреднять и производить анализ замеров стволов нефтяных скважин,
на основании данных, полученных с измерительного оборудования. Для достижения поставленной цели необходимо решить следующие задачи:
\begin{itemize}
  \item провести анализ существующих программных продуктов;
  \item разработка функциональной и информационной моделей, программного обеспечения;
  \item разработка модуля импорта данных
  \item разработка системы управления содержимым БД и усреднения замеров
  \item разработка модуля визуализации замеров
  \item разаботка модуля рассчетов расстояний между стволами
  \item разработка модуля визуализации расстояний между стволами
\end{itemize}

Формальной постановке задачи соответствует контекстная диаграмма методологии IDEF0, описывающая входные и выходные данные, управляющие воздействия и механизмы,
влияющие на систему в целом, приведенная на рисунке 1.1.:

%TODO: Добавить картинку и проверить нумерацию

\subsection{Структура решения задачи разработки программного обеспечения}

\subsubsection{Функциональная  модель}
%TODO: Добавить картинку и описание

\subsubsection{Информационная модель}
%TODO: Добавить картинку и описание

\subsection{Обзор технологий разработки программного обеспечения}

\subsubsection{Язык C++}
На данный момент, C++ остается одним из самых популярных и производительных языков программирования и применяется практически во всех прикладных областях
программирования, от низкоуровневого программирования для микроконтроллеров, до высокопроизводительных серверных приложений и компьютерных игр.

\subsubsection{SQLite}
SQLite — это встраиваемая кроссплатформенная СУБД, которая поддерживает достаточно полный набор команд SQL и доступна в исходных кодах (на языке C). На данный
момент является самой популярной встраиваемой СУБД. Применяется как на персональный компьютерах, так и в мобильных ОС и "умных" телевизорах.

\subsubsection{Qt}
Qt — кроссплатформенный инструментарий разработки ПО на языке программирования C++, доступен в исходных текстах. Позволяет создавать кросс-платформернные приложения с богатыми возможностями
графического интерфейса, работой с сетью, мультимедиа, БД и 3D-графикой. В окружении каждой поддерживаемой ОС будет выглядеть максимально похоже на "родные" приложения
системы.

\subsubsection{MathGL}
MathGL — кроссплатформенная библиотека для визуализации данных. Имеет интеграцию с Qt.

\subsubsection{Обоснованность выбора технологий}
На данный момент указанные технологии являются единственным способом, как выполнить требования о кроссплатформенности, так и получить легкий в поддержке
продукт, базирующийся на надежных и поддерживаемых библиотеках.
