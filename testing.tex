Тестирование является важной и обязательной частью процесса разработки.

Процесс тестирования можно разделить на 3 этапа:
\begin{itemize}
  \item проверка в нормальных условиях;
  \item проверка в экстремальных условиях;
  \item проверка в исключительных ситуациях.
\end{itemize}

\textbf{Тестирование в нормальных условиях}

При проверке в нормальных условиях программа функционировала соответствующим образом:
введённые данные были без потерь сохранены в базе данных в нужном формате и в результате запросов были выданы верные сведения.
\putimage{manual/3.jpg}{Ввод корректных параметров подрядчика}
\putimage{manual/4.jpg}{Созданный подрядчик}

\textbf{Тестирование в экстремальных условиях}

Проводилась проверка на ввод нулевых и отсутствующих параметров. Программа не позволяет ввести неверные значения, т.н.
"защита от дурака" (Рис. 26).

\putimage{tests/1.png}{Недоступная кнопка ОК при попытка создать заказчика без названия}

\textbf{Тестирование в исключительных ситуациях}

Тестирование устойчивости программы при вводе неверных данных проводилось с самого начала разработки.
Построение интерфейса программы предусматривает предотвращение возможности совершения пользователем действий,
приводящих к исключительным ситуациям.

Практически невозможна ситуация, когда в результате сбоя разработанное ПО выйдет из-под контроля и
нарушит целостность исходных данных, системы или других прикладных программ.

\textbf{Анализ тестирования}

Тестирование, проведённое в различных условиях, подтверждает работоспособность программы. Возможно, в процессе эксплуатации программы потребуются некоторые её доработки.
